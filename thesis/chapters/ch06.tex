This work explores the prospects of deep recurrent end-to-end architectures applied to speech recognition. Complementary aspects of developing speech recognition systems are eliminated by focusing on end-to-end speech units as a two-step process requiring a Connectionist Temporal Character Classification (CTCC)\cite{graves2006connectionist} model and Language Model (LM) rather than a three-step process requiring an Acoustic model(AM), LM and phonetic dictionary. A two-step process rather than a three-step process is particularly desirable for low resource languages as less effort is required developing fewer simplified models.

Earlier in chapter one, deep learning was defined as a type of representational learning whereby different levels of complexity are captured in internal layer-wise encapsulations. It has also been noted that layer-wise stacking of neural and neural network type architectures such as the Restricted Boltzmann Machine (RBM) deep belief networks (DBMs) are used to implement such representations. In this chapter, the end-to-end Bi-directional Recurrent Neural Network model is described. Here, the development of the features using the deep scattering convolution network is first elaborated on. The model parameters and architecture is described and the decoding algorithm is detailed.

\section{Deep Scattering Features}
The fast wavelet transformed is derived in Chapter 4 from a low pass filter and a high pass filter.  The speech features used in this research using a deep scattering network 2 layers deep was created using the wavelet modulus operator comprising a low pass filter and a band pass filter.   Hyper parameters of the system included the window period for each sampled sub section, T;  The Q-band value for the band pass filter and the number of wavelets $$J$$ at each scattering layer for the total number of layers, $$M=2$$.

The matlab scatnet toolbox \citep{anden2014scatnet}, used to determine the scatter coefficient features for this research, provides optimal values for hyper parameters for audio signal processing into scatter features.  In this regime the value for the hyper parameter $T$, the number of samples per window, = $512$ samples per window. This approximates a window of $50 milliseconds$ for the audio signals sampled at $8000 Hz$.  For the first scattering layer the parameter, $Q=8$ and for the second scattering layer, the $Q=1$.  Finally $J$ is pre-calculated based on the value of $T$.  These after scatnet processing, eventually produce a feature-vector $165$ coefficients long.  These feature vectors in turn are used as inputs to the bi-direction neural network model whose architecture is explained in the succeeding sections.

\section{CTCC-BiRNN Architecture}
The core of the system is a bidirectional recurrent neural network (BiRNN) trained to ingest scatter coefficients described in the previous section, in order to generate English text transcriptions.  An end-to-end system therefore specifies that utterances $x$ and the corresponding label $y$ be sampled from a training set such that the sample $S = {(x^{(1)}, y^{(1)}), (x^{(2)}, y^{(2)}), . . .}$.   In our end-to-end model, each utterance, $x^{(i)}$ is a time-window consisting of $T^{(i)}$ samples.  Each window passes through a scattering transform to yield an input of vector of p features; so   $x^{(i)}_{t,p}$ denotes the $p$-th feature in a scatter transform at time .  

GPU training of the speech model architecture developed above was done using Mozilla deepspeech \cite{mozilla/deepspeech_2019} CTC bi-directional RNN implementation along with the accompanying Mozilla Common voice dataset  \cite{common voice by mozilla_2019}.  The Common Voice Dataset project consists of voice samples in short recordings approximately $4$ seconds each.  The complete dataset is about $250$ hours of recording divided into training, test and development subsets.  The BiRNN, given the input sequence, $x$, outputs a sequence of probabilities $y_t=\mathbb{P}(c_t|x)$,  where $c_t \in a,b,c,\dots,z,space,apostrophe,blank$. 

The actual architecture of our core Bi-RNN is similar to the deepspeech system described in \cite{hannun2014deep}. This structure consists of a 5 hidden layers and one output layer.  The first three layers are regular DNNs followed by a bi-directional recurrent layer. As such, the output of the first three layers are computed by:
\begin{equation}
    h^{(l)}_t = g(W^{(l)} h^{(l−1)}_t + b^{(l)})\label{ch06_01_l1-3}
\end{equation}

$g(\cdot) = min\{max\{0,z\},20\}$  is the clipped rectified linear unit and $W^{(l)},b^{(l)}$ are weight matrix and bias parameters for layer  as described in sections \ref{dnn} and \ref{deepspeech} respectively.

It was shown in chapter \ref{ch3RNN} the recurrent layer comprise a forward and backward RNNs whose equations are repeated here for reference
\begin{equation}
    h^{(f)}_t = g(W^{(4)} h^{(3)}_t + W^{(f)}_r h^{(f)}_{t−1} + b^{(4)})
    \label{ch06_02_fwd}
\end{equation}
\begin{equation}
h^{(b)}_t = g(W^{(4)} h^{(3)}_t + W^{(b)}_r h^{(b)}_{t+1} + b^{(4)})    \label{ch06_03_bwd}
\end{equation}

Consequently, $h^{(f)}$ is the sequential computation from $t=1$ to $t=T^{(i)}$ for the $i$-th utterance and $h^{(b)}$ is the reverse computation from $t=T^{(i)}$ to $t=1$.  In addition the output from layer five is summarily given as the combined outputs from the recurrent layer:
\begin{equation}
h^{(5)} = g(W^{(5)} h^{(4)} + b^{(5)})    \label{ch06_04_l5}
\end{equation}
where $h^{(4)} = h^{(f)} + h^{(b)}$. The output of the Bi-RNN on layer 6 is a standard soft-max layer that outputs a predicted character over probabilities for each time slice $t$ and character $k$ in the alphabet:
\begin{equation}
h^{(6)}_{t,k} = \hat{y}_{t,k} \equiv \mathbb{P}(c_t = k \mid x) = \frac{\exp{ \left( (W^{(6)} h^{(5)}_t)_k + b^{(6)}_k \right)}}{\sum_j \exp{\left( (W^{(6)} h^{(5)}_t)_j + b^{(6)}_j \right)}})    \label{ch06_05_l6}
\end{equation}

$b^{(6)}_k$ takes on the -th bias and $(W^{(6)} h^{(5)}_t)_k$ is the matrix product of the $k$-th element.  The error of the outputs are then computed using the CTC loss function \cite{graves_2014} described in chapter \ref{ch3DNN}.  A summary of our model is illustrated in figure \ref{ch06_01_ctc_scatter}.
