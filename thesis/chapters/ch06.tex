
\textcolor{blue}{
The details of the language models developed for the Wakirike language is discussed in this chapter.  The language models developed draw  upon the premise that the grammar of a language is expressed in the character sequence pattern ultimately rendered in word sequences.  The two models developed in this chapter follow RNN implementations discussed in chapter \ref{ch3RNN}. }

\section{Data Preparation}
\startblue
A published version of the Wakirike New Testament Bible was obtained and used as the data source for RNN training of the language model.  There was no readily available soft or on-line copy of the Wakirike new testament bible. As such, the Wakirike New Testament Bible text corpus text was entered into the ASR system from the physical copy using a text editor to form a text corpus.  The complete corpus had a word count of 668,522 words and a character count of 6,539,176 characters. Following the k-fold cross validation process, \citep{geron2019hands}, the data set was then divided into 11 parts. Two parts dedicated for testing and validation and the remaining nine parts were used for training. As the validation set is not seen during training it can be used to keep track of how well the training is going and that it is not over-fitting the data by simply memorising it.

Preprocessing of the text corpus involved selecting a set of characters as the input feature set and removing all other characters not found in the input feature set.  The Unicode representations of the character set consisted of letters and punctuation marks.  These are one-hot encoded and batched for sequential input.  Neural network parameters which are not automatically determined through back propagation are called hyper-parameters.  These are usually experimentally determined and manually set while configuring the network.  A hyper-parameter for the language model RNN is the input sequence length.  For the language model built a 30 characters-long sequence length is chosen.  This  length is an average phrase sequence.  In these phrases, long-term character dependencies of words can be captured. At the same time, keeping the sequence length at this value, and not longer, will pose less of a burden on the computer system resources during parameter computations.  

Another hyper-parameter for training used was the batch size.  The batch size parameter determined how many 30-character sequences will be trained in parallel in order to speed up the training process.  Increasing the batch size also meant an increase in the size of the matrix multiplications being performed and therefore, the computing power system resource being demanded by the language model.  By experimenting with various batch sizes it was determined a batch size of 200 was suitable for training the language model with respect to the other training parameters.

\section{GRU RNN Architecture}
The modified LSTM RNN known as the Gated Recurrent Unit (GRU) discussed in Chapter \ref{ch3RNN} is employed for the neural network model built in this Chapter.  In order to optimise network performance while conserving computation resources, GRUs have been shown to give similar performance to regular LSTMs however, with a lighter system resource footprint \citep{cho2014learning}. 

The architecture of the GRU RNN used to train the Wakirike text corpus had an internal network size of 512 nodes for each layer and was 3 layers deep. In a study by \citep{goodfellow2013multi}, it was shown that increasing the number of nodes in a neural network will lead to over-fitting, however, simultaneously increasing the network depth mitigates this effect.  In other words, in order to expand the degrees of freedom of a neural network and at the same time constrain the network to generalise well on unseen data, it is necessary to increase the number of neurons in both length and depth.  Experiments carried out in this chapter follow this recommendation. Initial experiments had an internal node size of 128 and a single layer deep.  While this showed promise of converging, the error rate was still high, therefore the network was expanded to the final model above.   Externally, the network model is further  sequenced 30 times, representing the input sequence length hyper-parameter and the number of recurrent connections where each connection represents a sequenced time step. 

Another hyper-parameter sensitive to network size is the learning rate.  The learning rate is selected in such a manner that an increase in the network size makes the learning rate more prone to overshooting.  Therefore, increased degrees of freedom in a neural network will require the learning rate to be made smaller so that it does not overshoot the network saturation point.  Small learning rates of between 0.001 and 0.005 were used. Furthermore, the language model neural network was designed to overcome over-fitting using the dropout method \citep{srivastava2014dropout} which has been shown to be effective for regularising deep neural networks.  The hyper-parameter for dropout was kept at 20\% such that only 80\% of neural network activations are propagated from one layer to the next, whereas the remaining 20\% were randomly zeroed out.  Intuitively, dropout works by forcing the remaining active neurons to infer what is missing in the activations that have been dropped and ultimately leads to better generalisations as activations are based on inference than on memory.

\section{Language Model Training Experiments}
Two sets of character-based neural network RNN-based experiments are developed in this chapter.  A third word based statistically modelled language model is also developed based on  \cite{Heafield-estimate} estimates as a baseline model. Character-based perplexity measurements were used to compare the character-based models and a conversion factor based on \citep{hwang2017character} is used to compare character-based models on the word-based counterparts.  Experiments for the RNN language models were majorly performed using tensorflow-MKL, which is a highly parallelised (44-threads for one of the experiments) cpu-based experiments.  The experiment with the largest number of neurons was also performed on cloud-based GPUs (Nvida Tesla T4).  Details of the experiments carried out and resulting perplexity are shown in Table \ref{tab6_1:LMX}.

\begin{table}
  \caption{Language Models comparison}
  \label{tab6_1:LMX}
\begin{tabular}{lrrr}
\toprule
Language Model & Train time & Perplexity & Epochs  \\
\midrule
GRU RNN 3-layer model (CPU training) & ~ 2 days & 30.920 & 75 \\
5-gram with Keysner Soothing and interpolation & 5 minutes & 238.720 & N/A \\
GRU RNN single-layer model (GPU training) & ~ 5 hours & $2.35\times10^{13}$ & 120 \\
Plain RNN single-layer model (GPU training) & ~ 9 hours & $2.35\times10^{13}$ & 180 \\
GRU RNN 3-layer model (cloud GPU training) & ~ 2 hours & 30.920 & 75 \\
\bottomrule
\end{tabular}
\end{table}

\stopblue


\section{Output Language Model and Language Generation}\label{sec6_4}
\startblue
The 3-layer network experiments were trained on both CPU and GPU configurations. Both were trained for 75 epochs, where an epoch indicates that the model has processed all of the training data.  Recall that the model  is trained until it is saturated.  In other words, the model trains until the prediction accuracy is no longer improving.   This usually will take several epochs.

The loss plots for the three-layer and single layer GRU RNN are shown in Figure \ref{fig_ch7_00losses}.  The three-layer GRU-RNN achieved a prediction accuracy of 65\% on held-out data.  When the created 3-layer GRU character-based RNN language model is seeded with an input character, one can force the network to select from the top-N candidates thus causing the Neural network to generate its own sentences.  In this scenario, the network is said to perform language generation by constructing its own sentences.  The generated language output from the GRU language model was found to be a reflection of the overall context of the training data. 

\begin{figure}
\centering
  % Requires \usepackage{graphicx}
  \includegraphics[width=14cm]{thesis/images/losses_slgru}\\
  \caption{Loss curves for a. 3-Layer GRU and b) Single-Layer RNN} \label{fig_ch7_00losses}
\end{figure}

The evaluation of the GRU language model of the Wakirike language was performed using a perplexity measurement metric. The Perplexity metric applies the language model to a test data-set and measures how probable the test data-set is. Perplexity is a relative measure given by the formula:
%
\begin{equation}
PP(W)=P(w_1,w_2\dots w_N)^\frac{1}{N}
\label{ch5_eq1_ppx}
\end{equation}
%
%
\begin{equation}
PP(W)=\sqrt[N]{\prod_{i=1}^N\frac{1}{P(w_i|w_{i-1})}}
\label{ch5_eq2_ppx}
\end{equation}
%
Where $w_1,\dots,w_N$ are the sequence of words. The language model with the lower relative perplexity score is therefore expected to yield better approximation of the data when applied to unseen data generally.

Intuitively, the perplexity metric measures the ability of a language model to predict held-out data.  A character based perplexity metric is possible using  the negative log likelihood of the character sequence.  A character based perplexity metric is possible using  the negative log likelihood of the character sequence.

\begin{equation}
    PP(X)=exp\left\{−\frac{\sum_{t=1}^T\log P(x_t|x_{1:t−1})}{T}\right\}
\label{ch5_eq3_ppx}
\end{equation}

However, our base-line language model is a 5-gram word-based language model.  Therefore, comparing a word based model to a character based model requires a conversion step. In this work, the conversion step involved using the GRU language model generated a corpus which was re-scored by re-estimating with a 5-gram word-based language model

The result of the training of the GRU-Cell Recurrent Neural Network on low-resourced Wakirike Language gave impressive and intelligible results and showed better results when measured with standard n-gram language models. The results showed that it is indeed possible to derive a language model using a GRU-cell RNN on a low resource character sequence corpus for the Wakirike language.
\stopblue

Table \ref{tab6_1:LMX} shows the Results of the Perplexity model of the LSTM Wakirike Language model and an equivalent 5-gram Language model with interpolation and Keysner smoothing \citep{chen1996empirical} for various lengths of the held-out data.
\startblue
\section{Discussion}
The result of the training of the 3-layer-deep GRU-Cell Recurrent Neural Network on low-resourced Wakirike Language had decent results with an accuracy of 65\%.  Even with this accuracy, the model had shown to have learned the vocabulary and was able to construct phrases consistent with the Wakirike language structure.  The single-layer GRU cell however after 120 epochs did not learn the vocabulary and was not able to learn any words.  It can also be seen from the loss curves of the language models that the models reach saturation fairly quickly after about 20 epochs for the 3-layer GRU RNN model and after about 40 epochs for the single-layer RNN.


The results also showed that the 3-layer GRU language model developed a better language model than the 5-gram model in terms of the perplexity metric because the perplexity of the 3-layer GRU RNN model was lower than that of the 5 gram model.  The single layer GRU model being a shallow model with a single-layer did not however learn anything having a  very high perplexity heading towards infinity.  The paragraphs below demonstrate output generations from the single layer and 3-layer GRU RNN language models based on the sampling procedure described in Section \ref{sec6_4}

\begin{table}
  \caption{Language Models sample generation}
  \label{tab6_1:LMX}
\begin{tabular}{lrrr}
\toprule
a) Original Wakirike Text \\
\midrule
\texttt{mine-o anikanika boro sobie korobo enjelapu so, we duko o piri sa ibiok-}\\
\texttt{wein mi sikima be jinye dukobia bo, ya tamuno worinime sime inibo piri} \\
\texttt{wa tatari duko borosam, a piki mioku bari ani dukoabe na nemikase tomonibo}\\
\midrule
GRU RNN 3-layer model (75 epochs) \\
\midrule
 \texttt{ani se mi be chinmgbolu mi ani se chua yee anisiki ini tamuno be bu s-}\\
\texttt{arame, se nwo beme, a kokomaye duko o piriabe, o bi se mi mieari ye mi}
\texttt{ori oria koki a kuro mi nyana yee. o bi bara mi o nwose o diebia  ani}
 \midrule
Plain RNN single-layer model (180 epochs) \\
\midrule
\texttt{min on o o bo oeuemin on o oniaia a bire nami bieee mani o onuo o be} 
\texttt{bo oe berimini okuma ani mani o o onuaminiana bireme,eanaminianiania b-}
\texttt{i bo ono bo onia anaa beremanaa bi nao sike,einama nieiei mi niei ia }
\midrule
GRU RNN single-layer model (120 epochs) \\
\midrule
\texttt{ia iiiii  o i ii i i iiii  iii i oi oi o oiai  oi ii ii ii  iiiii  i}
\texttt{iiii iiiiii  iii iii iii  ii iii  o oi i i o ii iiii  iiii  iiiii i ii}
\texttt{i iii i o ii oi oia  oi iiiiii  o i o o o i oi o oi iiiii i iiiii  ii}
\bottomrule
\end{tabular}
\end{table}



It would be in the interest of this research to further consider increasing the number of layers to achieve higher levels of accuracy or consider optimising other parameters which will help the training results such as the sequence length of the model.  However, this was not done as the time constraint of one day for training the language model was already stretched and we are certain that these hyper-parameters have a direct influence on the size of the model.  This in turn will affect the computing resources required and hence a trade-off of the training time required to saturate the models.
\stopblue

\section{Chapter Summary}
This chapter shows the application of a character-based Gated Recurrent Unit RNN on the low resource language of Wakirike to generate a language model for the Wakirike language.  The data-set and preparation and the details of the network were discussed.  The output of this model was used to hallucinate the Wakirike language which was then scored against word-based perplexity to obtain a metric against the baseline language model.

It can be inferred that the GRU character-model developed has an improved language model and because it is based on a character-model, which is fine-grained when compared to a word model, it is likely to generalise data better when used in practice and is less biased than a word-based model.  This can be observed from the fact that the output corpus produced a larger vocabulary size.
