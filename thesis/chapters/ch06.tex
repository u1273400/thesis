Neural networks have become increasingly popular due to their ability to model non-linear system dynamics. Since their inception, there have been many modifications made to the original design of having linear affine transformations terminated with a nonlinear functions as the means to capture both linear and non-linear features of the target system. In particular, one of such neural network  modifications, namely the recurrent neural network, has been shown to overcome the limitation of varying lengths in the inputs and outputs of the classic feed-forward neural network.  In addition the RNN is not only able to learn non-linear features of a system but has also been shown to be effective at capturing the patterns in sequential data.

A language model for the Wakirike language is developed in this chapter.  This model draws upon the premise that the grammar of a language is expressed in the character sequence pattern ultimately revealed in the words rendered by the character sequences.  Therefore, abstract grammar rules can be extracted and learned by a character-based RNN neural network.  Specialised implementations of the RNN called the Long Short Term Memory (LSTM) and also the Gated Recurrent Unit (GRU), discussed in chapter \ref{ch3RNN}, are designed to capture patterns over particularly long data sequences and are thus, ideal candidates for generating character sequences while preserving syntactic language rules in the words formed from generated character sequences.  These long-term relationships and patterns are learned by the neural network model from the training data it ingests.

\section{Data Preparation}
The Wakirike New Testament Bible served as the source of data for the deep neural network training.  There is no readily available soft or on-line copy of the Wakirike new testament bible. As such, the Wakirike New Testament Bible was typed to form a text corpus, giving rise to a complete corpus word size of 668,522 words and a character count of 6,539,176 characters.  The data set was then divided into 11 parts. Two parts dedicated for testing and validation and the remaining nine parts were used for training.

The Unicode representations of the character set consisting of letters and punctuation marks are one-hot encoded and batched for sequential input, each batch having a character sequence length of 30 characters.


\section{GRU Training}

The modified LSTM RNN known as the Gated Recurrent Unit (GRU) is employed for the neural network model built in this work, in order to optimise network performance while conserving computation resources.  GRUs have been shown to give similar performance to regular LSTMs however, with a lighter system resource footprint \citep{cho2014learning}. The GRU RNN used to train the Wakirike text corpus comprised an internal network size of 512 nodes for each layer and was 3 layers deep. Externally, 30 GRUs represented  the number of recurrent connections each connection representing a time step bearing contextual for the recurrent input sequence. 

To mitigate for over-fitting, due to the multi-layered high-dimensional depth of this neural network, a small learning rate of 0.001 was used. To further marginalise over-fitting the popular and effective dropout method \citep{srivastava2014dropout} for regularising deep neural networks was kept at 20\% such that only 80\% of neural network activations are propagated from one layer to the next, whereas the remaining 20\% were randomly zeroed out.

\section{Output Language Generation}
The neural network was trained for 10 epochs and achieved a prediction accuracy of 85\% on held-out data.  With this GRU character-based language model, it is possible to seed this network with an input character and select from the top-N candidates thus causing the Neural network to generate its own sentences.  In this scenario, the network is said to perform language generation by immanently constructing its own sentences.  The generated language output from the GRU language model was found to be intelligible and a reflection of the overall context of the training data.

A clever use of this new corpus generated by the GRU language model of this work was to determine a word-based perplexity metric for the GRU neural language model. In this work, the word-based perplexity metric was determined from the output language generated by first estimating the word based perplexity on the training data.  The same perplexity calculation was then used on the generated neural language model corpus. The corpus size of the neural language model was made to be equivalent to that of the training data, that is containing 6,539,176 characters.  The perplexity calculation was based on a modified Kneser-Key 5-gram model with smoothing \citep{Heafield-estimate}.  The results discussed below showed that the GRU-based RNN model generated a superior model compared to the n-gram model that better matched the training data.

The evaluation of the GRU language model of the Wakirike language was performed using a perplexity measurement metric. The Perplexity metric applies the language model to a test data-set and measures how probable the test data-set is. Perplexity is a relative measure given by the formula:
%
\begin{equation}
PP(W)=P(w_1,w_2\dots w_N)^\frac{1}{N}
\label{ch5_eq1_ppx}
\end{equation}
%
%
\begin{equation}
PP(W)=\sqrt[N]{\prod_{i=1}^N\frac{1}{P(w_i|w_{i-1})}}
\label{ch5_eq2_ppx}
\end{equation}
%
Where $w_1,\dots,w_N$ are the sequence of words. The language model with the lower relative perplexity score is therefore expected to yield better approximation of the data when applied to unseen data generally.

The result of the training of the GRU-Cell Recurrent Neural Network on low-resourced Wakirike Language gave impressive and intelligible results and showed better results when measured with standard n-gram language models. The results showed that it is indeed possible to derive a language model using a GRU-cell RNN on a low resource character sequence corpus for the Wakirike language.

A character based perplexity metric is possible using  the negative log likelihood of the character sequence.
\begin{equation}
    PP(X)=exp\left\{−\frac{\sum_{t=1}^T\log P(x_t|x_{1:t−1})}{T}\right\}
\label{ch5_eq3_ppx}
\end{equation}

However, our base-line language model is a 5-gram word-based language model.  Therefore, comparing a word based model to a character based model requires a conversion step. In this work, the conversion step involved using the GRU language model generated a corpus which was rescored by re-estimating with a 5-gram word-based language model

Table ~\ref{tab:example} shows the Results of the Perplexity model of the LSTM Wakirike Language model and an equivalent 5-gram Language model with interpolation and Keysner smoothing \citep{Heafield-estimate} for various lengths of the held-out data.


\begin{table}
  \caption{Perplexity Calculation results}
  \label{tab:example}
\begin{tabular}{lrr}
\toprule
Language Model & Perplexity  \\
\midrule
Held-out data size (characters) & 998 & 99\\
\midrule
LSTM RNN & 1.6398 & 1.7622\\
5-gram with Keysner Soothing and interpolation & 1.8046 & 1.9461\\
\bottomrule
\end{tabular}
\end{table}
\section{Chapter Summary}
This chapter shows the application of a character-based Gated Recurrent Unit RNN on the low resource language of Wakirike to generate a language model for the Wakirike language.  The data-set and preparation and the details of the network were discussed.  The output of this model was used to hallucinate the Wakirike language which was then scored against word-based perplexity to obtain a metric against the baseline language model.

It can be inferred that the GRU character-model developed has an improved language model and because it is based on a character-model, which is fine-grained when compared to a word model, it is likely to generalise data better when used in practice and is less biased than a word-based model.  This can be observed from the fact that the output corpus produced a larger vocabulary size.
