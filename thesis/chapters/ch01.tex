Automatic Speech Recognition is a subset of Machine Translation that takes a sequence of raw audio information and translates or matches it against the most likely sequence of text as would be interpreted by a human language expert.  In this thesis, Automatic Speech Recognition will also be referred to as 
ASR or speech recognition for short.

It can be argued that while ASR has achieved excellent performance in specific applications, much is left to be desired for general purpose speech recognition. While commercial applications like Google voice search and Apple Siri gives evidence that this gap is closing, there is still yet other areas within this research space that speech recognition task is very much an unsolved problem.

It is estimated that there are close to 7000 human languages in the world \citep{besacier2014automatic} and yet for only a fraction of this number have there been efforts made towards ASR.  The level of ASR accuracy that have been so far achieved are based on large quantities of speech data and other linguistic resources used to train models for ASR. These models which depend largely on pattern recognition techniques degrade tremendously  when applied to different languages other than the languages that they were trained or designed for  \citep{Rosenberg2017end,besacier2014introduction}. More specifically, the collection of sufficient amounts of linguistic resources required to create accurate models for ASR are particularly laborious and time consuming sometimes extending to decades \citep{goldman2011easyalign,stan2016alisa}.  Research, therefore, geared towards alternative approaches towards developing is ASR systems that are reproducible across languages lacking the resources required to build robust systems is apt.

\section{ASR As a Machine Learning  problem}\label{ASRMLP}
\pagestyle{plain}
Automatic speech recognition can be put into a class of Machine Learning problems described as sequence pattern recognition because an ASR attempts to discriminate a pattern from the sequence of speech utterances. 

One immediate problem realised with this definition leads us to discuss statistical speech models that address how to handle the problem described in the following paragraph.

Speech is a complex phenomena that begins as a cognitive process and ends up as a physical process \citep{becchetti1998}.  The process of automatic speech recognition attempts to reverse engineer steps back from the physical process to the cognitive process giving rise to latent variables or mismatched data or loss of information from interpreting speech information from one physiological layer to the next.

It has been acknowledged in the research community \citep{2015watanabe,deng2013machine}  that work being done in Machine Learning has enhanced the research of automatic speech recognition.  Similarly any progress made in ASR usually constitutes a contribution to enhances made in the Machine Learning field.  This also is an attribution to the fact that speech recognition is a sequence pattern recognition problem.  Therefore techniques within speech recognition could be applied generally to sequence pattern recognition problems.

The two main approaches to Machine Learning problems historically involve two methods rooted in statistical science.  These approaches are generative and discriminative models.  From a computing science perspective, the generative approach is a brute-force approach while the discriminative model uses a rather heuristic approach to Machine Learning. This chapter presents the introductory ideas behind these two approaches and establishes the motivation for the proposed models used in this research for low resource speech recognition, as well as introducing the Wakirike language as the motivating language case study.

\section{Generative-Discriminative Speech Models disambiguation}
\pagestyle{fancy}
In the next chapter, the Hidden Markov Model (HMM) is examined as a powerful and major driver behind generative modelling of sequential data like speech.  Generative models are data-sensitive models because they are derived from the data by accumulating as many different features which can be seen and make generalisations based on the features seen. The discriminative model, on the other hand, has a heuristic approach to form a classification.  Rather than using features of the data directly, the discriminative method attempts to characterise the data into features. It is possible to conclude that the generative approach uses a bottom-to-top strategy starting with the fundamental structures to determine the overall structure, while the discriminative method uses a top-to-bottom approach starting with the big picture and then drilling down to discover the fundamental structures.

Ultimately, generative models for Machine Learning learning can be interpreted mathematically as a joint distribution that produces the highest likelihood of outputs and inputs based on a predefined decision function.  The outputs for speech recognition being the sequence of words and the inputs for speech being the audio waveform or equivalent speech sequence.

\begin{equation}
d_y(\mathbf{x};\lambda)=p(\mathbf{x},y;\lambda)=p(\mathbf{x}|y;\lambda)p(y;\lambda)
\label{eqn1_1}
\end{equation}

where $d_y(\mathbf{x};\lambda)$ is the decision function of $y$ for data labels $\mathbf{x}$.  This joint probability expression given as $p(\mathbf{x}|y;\lambda)$ can also be expressed as the conditional probability product in equation (\ref{eqn1_1}).  In this equation, $\lambda$ predefines the nature of the distribution \citep{deng2013machine} referred to as model parameters.

Similarly, Machine Learning discriminative models are described mathematically as the conditional probability defined by the generic decision function below:
\begin{equation}
d_y(\mathbf{x};\lambda)=p(y|\mathbf{x};\lambda)
\label{eqn1_2}
\end{equation}

It is clearly seen that the discriminative paradigm follows a much more direct approach to pattern recognition. Although this approach appears cumbersome to model, this research leans towards this direct approach.  However, what the discriminative model gains in discriminative modularity, it loses in the model parameter estimation of ($\lambda$) in equation  (\ref{eqn1_1}) and (\ref{eqn1_2}).  
 As this research investigates, although the generative process is able to generate arbitrary outputs from learned inputs, its major drawback is the direct dependence on the training data from which the model parameters are learned. Specific characteristics of various Machine Learning models are reserved for later chapters, albeit the heuristic nature of the discriminative approach, which means not directly dependent on the training data,  gains over the generative approach as discriminative models are able to better compensate for latent variables\cite{gales2012structured}.  
 
 In the case of speech signals, information is lost in training data due to the physiologic transformations of the intended speech message as it moves from one speech mechanism mentioned in section \ref{ASRMLP} to the next.  The theme of pattern recognition through arbitrary layers of complexity is reinforced in the notion of deep learning defined in \cite{deng2014deep} as an attempt to learn patterns from data at multiple levels of abstraction. Thus while shallow Machine Learning models like Hidden Markov Models (HMMs) define latent variables for fixed layers of abstraction, deep Machine Learning models handle hidden/latent information for arbitrary layers of abstraction determined heuristically.  As deep learning mechanisms are typically implemented using deep neural networks, this work applies deep recurrent neural networks as an end-to-end discriminative classifier for speech recognition.  This is a so called "end-to-end model" because it adopts the top-to-bottom Machine Learning approach. Unlike the typical generative classifiers that require sub-word acoustic models, the end-to-end models develop algorithms at higher levels of abstraction as well as the lower levels of abstraction.  In the case of the deep-speech model \citep{hannun2014first} utilised in this research, the levels of abstraction include sentence/phrase, words and character discrimination. A second advantage of the end-to-end model is that because the traditional generative models require various stages of modeling including an acoustic, language and lexicon, the end-to-end discriminating multiple levels of abstractions simultaneously only requires a single stage process, greatly reducing the amount of resources required for speech recognition.  From a low resource language perspective this is an attractive feature meaning that the model can be learned from an acoustic only source without the need of an acoustic model or a phonetic dictionary.  In theory this deep learning technique is sufficient in itself without a language model.  However, applying a language model was found to serve as a correction factor further improving recognition results \citep{hannun2014deep}. 

\section{Low Resource Languages}
A second challenge observed in complex Machine Learning models for both generative as well as discriminative learning models is the data intensive nature required for robust classification models. \cite{saon2015ibm} recommends around 2000 hours of transcribed speech data for a robust speech recognition system. As is covered in the next chapter, for new languages, which are low in training data such as transcribed speech, there are various strategies devised for low resource speech recognition. \cite{besacier2014automatic} outlines various matrices for bench-marking low resource languages.  From the generative speech model interest perspective,  reference is made to languages having less than ideal data in transcribed speech, phonetic dictionary and a text corpus for language modelling.  For end-to-end speech recognition models interests, the data relevant for low resource evaluation is the transcribed speech and a text corpus for language modelling.  It is worth noting that it was observed \citep{besacier2014automatic} that speaker-base often does not affect a language resource status of a language and was often observed that large speaker bases could in fact lack language/speech recognition resources and that some languages having small speaker bases did in fact have sufficient language/ speech recognition resources.

Speech recognition methods looked at in this work are motivated by the Wakirike language discussed in the next section, which is a low resource language by definition.  Thus this research looked at low research language modelling for the Wakirike language from a corpus of Wakirike text available for analysis.  However, due to the insufficiency of transcribed speech for the Wakirike language, English language was substituted and used as a control variable to study low resource effects of a language when exposed to speech models developed in this work.

\section{The Wakirike Language}
The Wakirike municipality is a fishing community comprising 13 districts in the Niger Delta area of the country of Nigeria in the West African region of the continent of Africa. The first set of migrants to Wakirike settled at the mainland town of Okrika between AD860 and AD1515 at the earliest.  These early settlers had migrated from Central and Western regions of the Niger Delta region of Nigeria.  As the next set of migrants also migrated from a similar region, when the second set of migrants met with the first settlers they exclaimed “we are not different” or “Wakirike” \citep{wakirike}.  

Although the population of the Wakirike community from a 1995 report \citep{ethnologue} is about 248,000, the speaker base is  significantly less than stipulated.  The language is classified as Niger-Congo and Ijoid languages.  The writing orthography is Latin and the language status is 5 (developing) \citep{ethnologue}.  This means that although the language is not yet an endangered language, it still isn't thriving and it is being passed on to the next generation at a limited rate.

The Wakirike language was the focus for this research.  And End-to-end deep neural network language model was built for the Wakirike language based on the availability of the new testament bible printed edition that was available for processing.  The corpus utilized for this thesis work is approximately 9,000 words.

\section{Research aims and objectives}
In this research we develop speech processing models and language models which deliver robust deep and recurrent neural network implementations towards low resource speech recognition.  In particular, we develop a language model based on Gated Recurrent Units (GRU) for the Wakirike language and a bi-directional recurrent neural network (Bi-RNN) speech model for the English language. 

\section{Main Contribution to knowledge}

This work contributes a neural language model for the low resourced language of Wakirike.  At the same time this work implements a unique combination of end-to-end deep recurrent neural network models with a pristine and state of the art audio signal processing mechanism involving a hierarchical scattering network to engineer features to compete with current acoustic and deep architectures for speech recognition.

\section{Thesis outline}
The outline of this report follows the development of an end-to-end speech recogniser and develops the theory based on the building blocks of the final system.  Chapter two introduces the speech recognition pipeline and the generative speech model.  Chapter two outlines the weaknesses in the generative model and describes some of the Machine Learning techniques applied to improve speech recognition performance. 

Various Low speech recognition methods are reviewed and the relevance of this study is also highlighted.  Chapter three describes Recurrent Neural Networks (RNNs). Starting with Multi-Layer Perceptrons (MLPs), we go on to specialised recurrent neural networks including long short-term memory (LSTM) networks and the Gated Recurrent Units (GRU) are detailed. These recurrent neural network units form building blocks of the language model for Wakirike language implemented in this work.

Chapter Four explains the wavelet theorem as well as the deep scattering spectrum. The chapter develops the theory from Fourier transform and details the significance of using the scattering transform as a feature selection mechanism for low resource recognition.  

Chapters five and six is a description of the models developed by this thesis and details the experiment setup along with the results obtained. Chapters seven is a discussion of the result and chapter 8 are the recommendations for further study. 


