\documentclass[12pt,twoside]{report}
\usepackage[utf8]{inputenc}
\usepackage{graphicx}
\graphicspath{ {images/} }
\usepackage[a4paper,width=150mm,top=25mm,bottom=25mm,bindingoffset=6mm]{geometry}
\usepackage{wrapfig}
\usepackage{lscape}
\usepackage{rotating}
\usepackage{epstopdf}

% package used by \citep and \citet
\usepackage[sort&compress,comma,authoryear]{natbib}
\usepackage[options ]{algorithm2e}

\usepackage{setspace}
\usepackage{amsmath}
\usepackage{booktabs}
\usepackage{fancyhdr}
\pagestyle{fancy}
\fancyhead{}
\fancyhead[RO,LE]{Deep Scattering and End-to-End Speech Models towards Low Resource Speech Recognition}
\fancyfoot{}
\fancyfoot[LE,RO]{\thepage}
\fancyfoot[LO,CE]{Chapter \thechapter}
\fancyfoot[CO,RE]{I. J. Alamina}
\renewcommand{\headrulewidth}{0.4pt}
\renewcommand{\footrulewidth}{0.4pt}

\title{
    {Deep Scattering and End-to-End Speech Models towards Low Resource Speech Recognition}\\
    {\includegraphics{university.png}\\
A thesis submitted to the University of Huddersfield in partial fulfilment of the requirements for the degree of Doctor of Philosophy    }
}


\author{Iyalla John Alamina}
\date{31 October, 2019}

\begin{document}

\maketitle
\spacing{1.5}

\chapter*{Abstract}
This thesis investigates and acknowledges the various limitations of Deep Neural Network (DNN) techniques when applied to low resource speech recognition.   Various aspects of developing corpora for speech recognition systems are explored.  In particular various recurrent neural network (RNN) techniques were explored to implement end-to-end speech and language models (LM). Gated Recurrent Units (GRU) RNNs were used employed for the language model for a low resourced Wakirike language while bidirectional recurrent neural networks (bi-RNNs) were used to create end-to-end speech recognition model for English language.

Previous systems employed for low resource speech recognition involving deep networks included various knowledge transfer mechanisms including hybrid hidden markov models (HMM) to deep neural networks (HMM-DNN) models and those that are HMM alone-based include subspace Gaussian Mixture Models (GMMs).   These models are based on the HMM generative model and N-gram language models.  However, the model developed in this thesis makes use of an end-to-end discriminative model using the Bi-RNN acoustic/speech model augmented using speech features from a specialised light weight convolution network-the deep scattering network (DSN).  While the light weight DSN helped to reduce the training complexity, at the same time by focusing on end-to-end with Connectionist Temporal Classification (CTC) decoding, the speech model was compressed into a one step process rather than a three-step process requiring an AM, LM and phonetic dictionary. The research therefore shows it is possible build speech recognition systems using this compact strategy, while, improving on speech features required for accurate speech pattern recognition by deploying deep scattering network features because they possess higher dimension as opposed to traditional speech features. 

\chapter*{Dedication}
To the praise and glory of our God and of His Christ.

\chapter*{Acknowledgement}
I thank my supervisory team headed by Dr David Wilson for the invaluable guidance and keen interest throughout my research.  

I also acknowledge my parents (Prof. Mrs. Jane Alamina and Dr. P. T. Alamina) for immense support shown.  My wife, children and family members have also stood by given and given all the encouragement I could ever need.  Thank you.  Finally, to all who have said a prayer and have contributed towards my studies or well being, I am grateful.

\chapter*{Copyright statement}
i.	The author of this thesis (including any appendices and/or schedules to this thesis) owns any copyright in it (the “Copyright”) and s/he has given The University of Huddersfield the right to use such copyright for any administrative, promotional, educational and/or teaching purposes.
ii.	Copies of this thesis, either in full or in extracts, may be made only in accordance with the regulations of the University Library. Details of these regulations may be obtained from the Librarian. This page must form part of any such copies made.
iii.	The ownership of any patents, designs, trademarks and any and all other intellectual property rights except for the Copyright (the “Intellectual Property Rights”) and any reproductions of copyright works, for example graphs and tables (“Reproductions”), which may be described in this thesis, may not be owned by the author and may be owned by third parties. Such Intellectual Property Rights and Reproductions cannot and must not be made available for use without the prior written permission of the owner(s) of the relevant Intellectual Property Rights and/or Reproductions 


\tableofcontents

\chapter{Introduction}
Automatic Speech Recognition is a subset of Machine Translation that takes a sequence of raw audio information and translates or matches it against the most likely sequence of text as would be interpreted by a human language expert.  In this thesis, Automatic Speech Recognition will also be referred to as 
ASR or speech recognition for short.

It can be argued that while ASR has achieved excellent performance in specific applications, much is left to be desired for general purpose speech recognition. While commercial applications like Google voice search and Apple Siri gives evidence that this gap is closing, there is still yet other areas within this research space that speech recognition task is very much an unsolved problem.

It is estimated that there are close to 7000 human languages in the world \citep{besacier2014automatic} and yet for only a fraction of this number have there been efforts made towards ASR.  The level of ASR accuracy that have been so far achieved are based on large quantities of speech data and other linguistic resources used to train models for ASR. These models which depend largely on pattern recognition techniques degrade tremendously  when applied to different languages other than the languages that they were trained or designed for  \citep{Rosenberg2017end,besacier2014introduction}. More specifically, the collection of sufficient amounts of linguistic resources required to create accurate models for ASR are particularly laborious and time consuming sometimes extending to decades \citep{goldman2011easyalign,goldman2011easyalign}.  Research, therefore, geared towards alternative approaches towards developing is ASR systems that are reproducible across languages lacking the resources required to build robust systems is apt.

\section{ASR As a Machine Learning  problem}\label{ASRMLP}
\pagestyle{plain}
Automatic speech recognition can be put into a class of machine learning problems described as sequence pattern recognition because an ASR attempts to discriminate a pattern from the sequence of speech utterances. 

One immediate problem realised with this definition leads us to discuss statistical speech models that address how to handle the problem described in the following paragraph.

Speech is a complex phenomena that begins as a cognitive process and ends up as a physical process \citep{becchetti1998}.  The process of automatic speech recognition attempts to reverse engineer the steps back from the physical process to the cognitive process giving rise to latent variables or mismatched data or loss of information from interpreting speech information from one physiological layer to the next.

It has been acknowledged in the research community \citep{2015watanabe,deng2013machine}  that work being done in Machine Learning has enhanced the research of automatic speech recognition.  Similarly any progress made in ASR usually constitutes a contribution to enhances made in the machine learning field.  This also is an attribution to the fact that speech recognition is a sequence pattern recognition problem.  Therefore techniques within speech recognition could be applied generally to sequence pattern recognition problems.

The two main approaches to machine learning problems historically involve two methods rooted in statistical science.  These approaches are the generative and discriminative models.  From a computing science perspective, the generative approach is a brute-force approach while the discriminative model uses a rather heuristic approach to machine learning. This chapter presents the introductory ideas behind these two approaches and establishes the motivation for the proposed models used in this research for low resource speech recognition, as well as introducing the Wakirike language as the motivating language case study.

\section{Generative-Discriminative Speech Models disambiguation}
\pagestyle{fancy}
In the next chapter, the Hidden Markov Model (HMM) is examined as a powerful and major driver behind generative modelling of sequential data like speech.  Generative models are data-sensitive models because they are derived from the data by accumulating as many different features which can be seen and make generalisations based on the features seen. The discriminative model, on the other hand, has a heuristic approach to form a classification.  Rather than using features of the data directly, the discriminative method attempts to characterise the data into features. It is possible to conclude that the generative approach uses a bottom-to-top strategy starting with the fundamental structures to determine the overall structure, while the discriminative method uses a top-to-bottom approach starting with the big picture and then drilling down to discover the fundamental structures.

Ultimately, generative models for machine learning learning can be interpreted mathematically as a joint distribution that produces the highest likelihood of outputs and inputs based on a predefined decision function.  The outputs for speech recognition being the sequence of words and the inputs for speech being the audio waveform or equivalent speech sequence.

\begin{equation}
d_y(\mathbf{x};\lambda)=p(\mathbf{x},y;\lambda)=p(\mathbf{x}|y;\lambda)p(y;\lambda)
\label{eqn1_1}
\end{equation}

where $d_y(\mathbf{x};\lambda)$ is the decision function of $y$ for data labels $\mathbf{x}$.  This joint probability expression given as $p(\mathbf{x}|y;\lambda)$ can also be expressed as the conditional probability product in equation (\ref{eqn1_1}).  In this equation, $\lambda$ predefines the nature of the distribution \citep{deng2013machine} referred to as model parameters.

Similarly, machine learning discriminative models are described mathematically as the conditional probability defined by the generic decision function below:
\begin{equation}
d_y(\mathbf{x};\lambda)=p(y|\mathbf{x};\lambda)
\label{eqn1_2}
\end{equation}

It is clearly seen that the discriminative paradigm follows a much more direct approach to pattern recognition. Although this approach appears cumbersome to model, this research leans towards this direct approach.  However, what the discriminative model gains in discriminative modularity, it loses in the model parameter estimation of ($\lambda$) in equation  (\ref{eqn1_1}) and (\ref{eqn1_2}).  
 As this research investigates, although the generative process is able to generate arbitrary outputs from learned inputs, its major drawback is the direct dependence on the training data from which the model parameters are learned. Specific characteristics of various machine learning models are reserved for later chapters, albeit the heuristic nature of the discriminative approach, which means not directly dependent on the training data,  gains over the generative approach as discriminative models are able to better compensate for latent variables\cite{gales2012structured}.  
 
 In the case of speech signals, information is lost in training data due to the physiologic transformations of the intended speech message as it moves from one speech mechanism mentioned in section \ref{ASRMLP} to the next.  The theme of pattern recognition through arbitrary layers of complexity is reinforced in the notion of deep learning defined in \cite{deng2014deep} as an attempt to learn patterns from data at multiple levels of abstraction. Thus while shallow machine learning models like hidden Markov models (HMMs) define latent variables for fixed layers of abstraction, deep machine learning models handle hidden/latent information for arbitrary layers of abstraction determined heuristically.  As deep learning are typically implemented using deep neural networks, this work applies deep recurrent neural networks as an end-to-end discriminative classifier, to speech recognition.  This is a so called "an end-to-end model" because it adopts the top-to-bottom machine learning approach. Unlike the typical generative classifiers that require sub-word acoustic models, the end-to-end models develop algorithms at higher levels of abstraction as well as the lower levels of abstraction.  In the case of the deep-speech model \citep{hannun2014first} utilised in this research, the levels of abstraction include sentence/phrase, words and character discrimination. A second advantage of the end-to-end model is that because the traditional generative models require various stages of modeling including an acoustic, language and lexicon, the end-to-end discriminating multiple levels of abstractions simultaneously only requires a single stage process, greatly reducing the amount of resources required for speech recognition.  From a low resource language perspective this is an attractive feature meaning that the model can be learned from an acoustic only source without the need of an acoustic model or a phonetic dictionary.  In theory this deep learning technique is sufficient in itself without a language model.  However, applying a language model was found to serve as a correction factor further improving recognition results \citep{hannun2014deep}. 

\section{Low Resource Languages}
A second challenge observed in complex machine learning models for both generative as well as discriminative learning models is the data intensive nature required for robust classification models. \cite{saon2015ibm} recommends around 2000 hours of transcribed speech data for a robust speech recognition system. As is covered in the next chapter, for new languages, which are low in training data such as transcribed speech, there are various strategies devised for low resource speech recognition. \cite{besacier2014automatic} outlines various matrices for bench-marking low resource languages.  From the generative speech model interest perspective,  reference is made to languages having less than ideal data in transcribed speech, phonetic dictionary and a text corpus for language modelling.  For end-to-end speech recognition models interests, the data relevant for low resource evaluation is the transcribed speech and a text corpus for language modelling.  It is worth noting that it was observed \citep{besacier2014automatic} that speaker-base often doesn't affect a language resource status of a language and was often observed that large speaker bases could in fact lack language/speech recognition resources and that some languages having small speaker bases did in fact have sufficient language/ speech recognition resources.

Speech recognition methods looked at in this work was motivated by the Wakirike language discussed in the next section, which is a low resource language by definition.  Thus this research looked at low research language modelling for the Wakirike language from a corpus of Wakirike text available for analysis.  However, due to the insufficiency of transcribed speech for the Wakirike language, English language was substituted and used as a control variable to study low resource effects of a language when exposed to speech models developed in this work.

\section{The Wakirike Language}
The Wakirike municipality is a fishing community comprising 13 districts in the Niger Delta area of the country of Nigeria in the West African region of the continent of Africa. The first set of migrants to Wakirike settled at the mainland town of Okrika between AD860 and AD1515 at the earliest .  These early settlers had migrated from Central and Western regions of the Niger Delta region of Nigeria.  As the next set of migrants also migrated from a similar region, when the second set of migrants met with the first settlers they exclaimed “we are not different” or “Wakirike” \citep{wakirike}.  

Although the population of the Wakirike community from a 1995 report \citep{ethnologue} is about 248,000, the speaker base is  much less than that.  The language is classified as Niger-Congo and Ijoid languages.  The writing orthography is Latin and the language status is 5 (developing) \citep{ethnologue}.  This means that although the language is not yet an endangered language, it still isn't thriving and it is being passed on to the next generation at a limited rate.

The Wakirike language was the focus for this research.  And End-to-end deep neural network language model was built for the Wakirike language based on the availability of the new testament bible printed edition that was available for processing.  The corpus utilized for this thesis work is approximately 9,000 words.

In this research we develop speech processing models and language models which deliver robust deep and recurrent neural network implementations towards low resource speech recognition.  In particular, we develop a language model based on Gated Recurrent Units (GRU) for the Wakirike language and a bi-directional recurrent neural network (Bi-RNN) speech model for the English language. 

\section{Main Contribution to knowledge}

This work contributes a neural language model for the low resourced language of Wakirike.  At the same time this work implements a unique combination of end-to-end deep recurrent neural network models with a pristine and state of the art audio signal processing mechanism involving a hierarchical scattering network to engineer features to compete with current acoustic and deep architectures for speech recognition.

\section{Thesis outline}
The outline of this report follows the development of an end-to-end speech recogniser and develops the theory based on the building blocks of the final system.  Chapter two introduces the speech recognition pipeline and the generative speech model.  Chapter two outlines the weaknesses in the generative model and describes some of the machine learning techniques applied to improve speech recognition performance. 

Various Low speech recognition methods are reviewed and the relevance of this study is also highlighted.  Chapter three describes Recurrent Neural Networks (RNNs). Starting with Multi-Layer Perceptrons (MLPs), we go on to specialised recurrent neural networks including long short-term memory (LSTM) networks and the Gated Recurrent Units (GRU) are detailed. These recurrent neural network units form building blocks of the language model for Wakirike language implemented in this work.

Chapter Four explains the wavelet theorem as well as the deep scattering spectrum. The chapter develops the theory from Fourier transform and details the significance of using the scattering transform as a feature selection mechanism for low resource recognition.  

Chapters five and six is a description of the models developed by this thesis and details the experiment setup along with the results obtained. Chapters seven is a discussion of the result and chapter 8 are the recommendations for further study. 




\chapter{Low Resource Speech Models, End-to-end models and the scattering network}\label{c02}
\input{chapters/ch02}

\chapter{Recurrent Neural Networks in Speech Recognition}\label{ch3RNN}
\section{Speech Processing software and tools}
\subsection{Alisa}\label{c3sec_alisa}
Alisa tool is a lightly supervised sentence segmentation tool based on Voice Activity Detection (VAD) algorithms \citep{stan2016alisa}.  It is so called lightly supervised because it requires small amounts of training data.  Generally the tool was asserted to be optimised for sentence segmentation and offered assistance in the creation of new speech corpora in a language-independent fashion. 

The Alisa tool researchers deploy a two-step method for aligning speech, and claim performance up to 70\% imperfect transcriptions often found in online resources can be successfully aligned with a word error rate of less than 0.5\%.  This tool is therefore said to be suitable for development multilingual and under-resourced language aligned speech-corpora.

The motivation behind Alisa was to reduce the time and effort used to gather a large amount of large amounts of quality data as well as actively eliminate the domain knowledge required to phonetically transcribe speech data.  In addition, and as a bonus to achieving the first objective, is the ability to migrate speech technology fairly seamlessly from one language to another and therefore realise the rather tedious task of automatic transcription of a new language.

\subsubsection{Alisa Architecture}
The goal of automatic transcription of new language with low resource constraint is particularly valuable to this research and as such, it would be relevant to review the enhancements introduced to Alisa.  The two step-method consists of a GMM-based sentence level segmenter and also an iterative grapheme acoustic model used for alignment.  The sentence level GMM-based speech segmenter is used to automatically segment speech into utterances which as discussed earlier forms the basic unit of processing within any ASR system.  This attempts to relieve the researcher off the manual process of segmenting the continuous audio file manually. This process included a GMM-based voice activity detector trained from about 10 minutes of manually labeled data. The second step grapheme based acoustic model is supplemented with a highly restricted word network they referred to as a skip network.  Together an iterative acoustic modelling training procedure is formulated.  The method described required the initial training data and a minimal labelling procedure that involved simple letter to sound rules and inter-sentence silence segments to provide an orthographic transcript of the initial 10 minute recording data.  Therefore, this process is resource-effective because non-experts can also provide this data.  The actual alignment process made use of a grapheme level Viterbi decoder to drive the iteratively self-trained grapheme models.  The model architecture is shown in the figure below.
\begin{figure}
\centering
  % Requires \usepackage{graphicx}
  \includegraphics[width=9cm]{thesis/images/alisa}\\
  \caption{Original waveform input for auto-correlation}\label{fig_c3_alisa00}
\end{figure}

Figure 3-5 Alisa Overview (Stan et al., 2016)
Figure 3-5 above shows a block diagram of the steps involved in the alignment.  The method can be applied to any language with an alphabetic writing system, given the availability of speech resource and its corresponding approximate script.

We have earlier mentioned that the downside of using grapheme based acoustic model is the increase in the error margin or trade of accuracy.  Several steps were introduced in Alisa to minimise this margin, the chief being the introduction of a tri-grapheme acoustic model which is modeled after using context dependent triphones in traditional acoustic modelling.  Other methods used drop down the error margin include the use of discriminative training with the Maximum Mutual Information (MMI) criterion (Schluter & Ney, 2001; Bouchard & Triggs, 2004) and methods described in (Novotney & Schwartz, 2009; Lamel, Gauvain & Adda, 2002). Finally, it was observed that Alisa provided good alignment but was not fully featured. For instance it had no way of adding insertions and substitutions in the audio data not provided in the transcription.  Secondly, Alisa is restricted to languages that can utilise the english alphabet.

\section{Initial Experiments}
The experiments in the following sections describe initial experiments based on the initial study of a language learning companion before the research was narrowed down to a low resource speech recognition.  These preliminary experiments in addition to a preliminary Language Learning Survey helped to narrow down the Research to the specific speech processing task of Low Resource Automatic Speech Recognition (LR-ASR).

The following sections describe analysis of raw wave-forms using auto-correlation signal processing in Matlab and experiments made with the Nao robot speech processing engine and experiments with speech recognition toolkit and speech processing tasks.  These tasks include digit recognition systems using CMUSphinx and Kaldi speech recognition toolkits and speech alignment tasks using Alisa tool.

\subsection{Auto-correlation Experiments}
Preliminary experiments were carried out on raw speech signals in an attempt to quickly segment individual phonemes based on a basic threshold algorithm.  Further experiments designed an autocorrelation algorithm to attempt to discover a phoneme alphabets in a particular dataset in a semi-supervised fashion.
 
This method had desirable goals when compared with other segmentation techniques outlined in the previous chapter. The chief being the ability to simulate a posterior distribution statistic from auto-correlation estimate.  This presents an unnormalised posterior distribution measurement of every phoneme segments over the entire signal.

The correlation theory is based on the idea that when a signals is superimposed on itself in a time-shifted manner, the convolution over itself is highest when the two signals have zero time lag that is, perfectly overlapped in sync and the better the overlapping the higher the value of the correlation and the lesser the signals are matched they tend to cancel out each other and hence a very low value of the correlation.  The normalised auto-correlation value is obtained in \cite{picone1996fundamentals} from a signal $x(n)$ in the following equation:
\begin{equation}
    \Psi(i)=\frac{\sum_{n=0}^{N-1}x(n)x(n-i)}{\left(\sum_{n=0}^{N-1}x(n)^2\right)\left(\sum_{n=0}^{N-1}x(n-i)^2\right)}\label{c3eq_corr}
\end{equation}
Based on experimental procedure, estimated locations of similar wave-forms representing segmented phonemes are calculated.  Although the procedure is subject to degrade in the face of most of the difficulties associated with dealing with raw audio waveform, it further emphasises the need for accurate speech features and pre-processing highlighted in the previous chapter.

This two stage procedure performs segmentation of phonemes and then discovery of phoneme clusters using a statistical auto-correlation algorithm.  The process is described in the following sections.
\begin{figure}
\centering
  % Requires \usepackage{graphicx}
  \includegraphics[width=9cm]{thesis/images/corr}\\
  \caption{Original waveform input for auto-correlation}\label{fig_c3_exp01}
\end{figure}

\subsubsection{Segmentation}
Figure \ref{fig_c3_exp02} describes the various steps of the segmentation phase while Figure \ref{fig_c3_exp01} shows the original audio file. At the segmentation phase, we first of all adjust the scale of the original raw audio file to have only positive values rather than having it centred on zero (Figure \ref{fig_c3_exp02}a).  At the next step a smoothing filter based on experimentation is used to perform both smoothing as well as determining the peaks and trough (Figure \ref{fig_c3_exp02}b).  Then a threshold is applied to segment the waveform based on discovered inflection points (Figure \ref{fig_c3_exp02}c).  
\begin{figure}
\centering
  % Requires \usepackage{graphicx}
  \includegraphics[width=9cm]{thesis/images/corr00}\\
  \caption{Original waveform input for auto-correlation}\label{fig_c3_exp02}
\end{figure}
\subsubsection{Auto-correlation}
At the auto-correlation stage estimated phoneme segment boundaries are stored in an array and cross-correlated with the original signal.  Even though at a top-level view, the entire signal is auto-correlated, at the individual segment level, the signals are cross correlated against one another.  Furthermore, to achieve a ‘fair’ correlation estimate, individual segments representing estimated phonemes need to be re-sampled to eliminate mismatching of contour representations of the individual phonemes.

The proposed auto-correlation algorithm performs both top-down and bottom-top processing.  In the first stage it does bottom-top segmentation, while in the second phase top-bottom auto-correlation.   The major weakness is this auto-correlation method the segmentation algorithm, data filtering and the feature representation.  The Bayesian method of segmentation \citep{kamper2016unsupervised} which is related to this method also improved on these weaknesses was able to improve on these weakness by using ASR feature preprocessing and a combination of acoustic embedding Dynamic Time Warping (DTW) for clustering rather than auto-correlation.  In effect using the extracted features for clustering is in theory a better speech estimate with less intrinsic noise for classification than using an only smoothed audio data.

\subsection{Experiments with Nao robot}
Nao is a humanoid robot developed mainly for deployment in environments for robotics education and development purposes.  Nao comes with a speech recognition software that offers features such as language settings and recognition sensitivity.  However it was understandably found to be limited because the Nao robot itself does not possess the processing power to perform CPU intensive training of acoustic models.  The now robot did however offer a level of support for using the pocketsphinx system. The pocketsphinx system is the C-language equivalent of CMUSphinx speech recognizer system also by Carnegie Mellon University.  Using the pocketsphinx method, acoustic models trained high performance systems can then be deployed to Nao for fast decoding within the Nao.  

\subsection{Digit Speech Recognition and Alignment Experiments}\label{sec_digitspeech}
These experiments were performed using CMU Sphinx4 recognition system and Kaldi speech recognition software.  While CMU Sphinx and pocketsphinx delivered standard interface for speech recognition using generative hybrid models, Kaldi speech in addition also offered advanced methods such as subspace Gaussian mixture model used to develop cross-lingual acoustic models and deep architectures for hybrid generative-discriminative models for speech recognition.   The main challenge with Kaldi was that it was CPU intensive and required a reasonable amount of parallel processing to achieve good results within a reasonable time period. 

Speech alignment experiments were performed using the Alisa \cite{stan2016alisa} tool which is a python based tool with calls made to the HMM toolkit \cite{young2002htk}.  The Alisa tool alignment process undergoes a semi-supervised process and requires an error prone time-intensive manual pre-alignment procedure.  The tool itself was found to be quite unstable and the output results were not very easily reproducible for further tests to be carried out on different datasets.  In addition to the time-intensive pre-alignment procedure made the tool not very useful for this research. Had the tool been more successful, the tool, which utilises Voice Activity Detection (VAD) algorithms, would have been especially useful for sentence segmentation of long sequences of transcribed audio speech.  This tool however still lacked in alignment at either a word-level or sub-word level of alignment required in ASR pipelines.


\section{End-to-end Research Experiments}\label{sec_postalign}
The main challenges of speech recognition using HMM-based toolkits such as Kaldi, is the requirement for aligned speech.  In more recent endeavours, there has been efforts towards automatic alignment of transcribed audio speech recordings through successive Baum-Welch estimation techniques \cite{gales2014speech,ragni2018automatic,ragni2014data}. However, this technique is not particularly compatible with end-to-end goals adopted for this research as it would require preprocessing and successive pre-training of the data set.

The following section describes the post-alignment experiments and in a later Chapter, how these methods deal with the problem of automatic speech alignment in a fashion which was compatible with end-to-end speech processing.  The end-to-end requirements were desirable for low-resource speech recognition as it introduces a simpler speech model design.  The downside however to the end-to-end approach is the dependency on very deep recurrent neural network structures which require large volumes of data for successful training.
\subsection{Tensor flow sequence-to-sequence character-to-diacritically-labelled-character model}\label{sec_c2d}
Experiments performed in this and the next three sections are all based on sequence-to-sequence modelling using recurrent neural networks. While the this and the following section represents precursor experiments around speech recognition tasks, the later two sections represent the final experiments reported in this work.

The character-to-diacritically labelled character model was a sequence-to-sequence diacritically labeled experiment to automatically infer diacritic transcriptions of the Wakirike language given the plain unmarked Wakirike language text as input.  This is a task, when achieved successfully is a sub task towards developing a phonetic dictionary for the Wakirike Language which in turn can be used in HMM speech recognition or equivalent  end-to-end models.  This experiment was a precursor experiment, the results of which were reserved for further study.
\subsection{Sequence-to-sequence Grapheme-to-Phoneme (G2P) model}
This is a follow up experiment to the previous experiment in section \ref{sec_c2d}. This model attempts to automatically generate a phonetic dictionary from graphemes in a text corpus. Grapheme-to-phoneme experiments come in two flavours. The first being a continuation of the previous experiment, that is, using diacritically marked symbols and the second flavour using non-marked graphemes as input.  The experiments we performed used the latter non-marked graphemes as input. As this experiment was also a subtask in HMM speech model building,  the results of these experiments were reserved for further study.  

What follows in the next three sections are sequence-to-sequence experiments actively developed in this research and are detailed in chapters (\ref{ch6_wlm,ch6_speech,ch8_future}).  A brief summary of the experiments are highlighted in the following sections (\ref{sec_grulm,sec_be2e,c3sec_espnet}).  Note that these models all utilise TensorFlow deep learning library including the Bi-directional speech model (section \ref{sec_be2e}) which is built on top of Mozilla DeepSpeech with the exception of section \ref{c3sec_espnet} which is based on pytorch; a similar deep learning library (see table \ref{tab_tfstats} for comparison).

\subsection{GRU language model for Wakirike language based on TensorFlow}\label{sec_grulm}
The language model developed in this research is a character-based sequence-to-sequence deep recurrent neural network that maps a sequence of characters to a sequence of words found in the training data set. This model met the objective of reducing the vocabulary size required for language models as well as the text corpus required as inferences could be made over the smaller-fixed character vocabulary rather than orders or magnitude larger word corpus with the possibility of out of vocabulary terms found in the training data.  Though this may occur in the character sequence-model at the inference stage.  It would not normally happen during training.  The neural network model developed is described in Chapters \ref{}, consists of Gated Recurrent Unit (GRU) Recurrent Neural Network (RNN). The GRU is a specialised type of Long Short-Term Memory (LSTM) cell RNN.  The emphasis here is on the ability to model over particularly long sequences of the training data.  In this case, over long character sequences.  Thus, the network is able to learn long term dependencies as would be naturally required to construct grammatically correct sentences.  In essence, the RNN is able to learn grammar rules inherently from the training data.

\subsection{Bi-Directional LSTM-based end-to-end speech model}\label{sec_be2e}
A similar LSTM sequence-to-sequence network based on Baidu Research’s original research design \citep{hannun2014deep} is developed in this research for end-to-end speech recognition.  This model, as its name implies, attempts to establish long term relationships by adding a reinforcing LSTM layer learning information but this time from the opposite direction, hence the bi-directional architecture.  

In addition, the model incorporates the Connectionist Temporal Classifier (CTC) decoder. This enables the model to make run-time inferences on both the character as well as estimate audio wave to character label alignment simultaneously.  This makes this design accommodate end-to-end goals and ultimately simplifies the overall design and completely eliminates the need for either manual or semi-supervised alignments mentioned previously in sections (\ref{sec_alisa,sec_digitspeech,sec_postalign}).

\subsection{ESP-Net Experiments}\label{c3sec_espnet}
The ESP-Net (End-to-end Speech Network) toolkit \citep{watanabe2018espnet}, is a speech processing toolkit that was of interest to this research because it offers end-to-end capabilities not only in Automatic Speech Recognition (ASR) but also in Text-To-Speech (TTS) or speech synthesis and other speech-sequence-processing related tasks.  In addition, the toolkit offered multi-modal training combining both Attention networks \cite{vaswani2017attention} with CTC networks as well as multi-channel feature representation that is, the fusing together of multiple feature representations of data.  Only preliminary experiments were carried out using ESPNet and is discussed in Chapter \ref{ch08furtherstudy} of this work.

\section{Method of evaluation}
System building methodology \citep{nunamaker1990systems} for speech recognition systems requires models to be evaluated against speech recognition Machine Learning metrics.  For language models, perplexity metric was used for evaluation.  Bleu has also been used as a metric for evaluating language models.

Perplexity measures the complexity of a language that the language model is designed to represent \citep{1976jelinekcontinuous}. In practice, the entropy of a language with an N-gram language model $P_N(W)$ is measured from a set of sentences and is defined as
\begin{equation}H=\sum_{\mathbf{W}\in\Omega}P_N(\mathbf{W})
\label{eqn_c2_lm05}
\end{equation}

where $\Omega$ is a set of sentences of the language. The perplexity, which is interpreted as the average word-branching factor, is defined as
\begin{equation}PP(W)=2^H
\label{eqn_c2_lm06}
\end{equation}
where H is the average entropy of the system or the average log probability defined as
\begin{equation}
H=-\frac{1}{N}\sum_{i=1}^N[log_2P(w_1,w_2\dots w_N)]
\label{eqn_c2_lm07}
\end{equation}
For a bi gram model therefore, equation (\ref{eqn_c2_lm07}) becomes
\begin{equation}
PP(W)=2^H=2^{-\frac{1}{N}\sum_{i=1}^N[log_2P(w_1,w_2\dots w_N)]}
\label{eqn_c2_lm08}
\end{equation}
After simplifying we have
\begin{equation}
PP(W)=\sqrt[N]{\prod_{i=1}^N\frac{1}{P(w_i|w_{i-1})}}
\label{eqn_c2_lm09}
\end{equation}


Full speech recognition pipelines are usually evaluated against the Word Error Rate (WER).  WER is computed as follows:
\begin{equation}\label{eqn_2_3_wer}
WER=\frac{I+D+R}{WC}\times 100
\end{equation}
Here $I,D,$ and $R$ are wrong insertions, deletions and replacements respectively and $WC$ is the word count.

Metrics used for low speech recognition in the zero speech challenge \citep{versteegh2015zero} include the ABX metric. Other common speech recognition error metrics following a similar definition as the Word Error Rate (WER) are Character Error Rate (CER), Phoneme Error Rate (PER) and Syllabic Error Rate (SyER) and sentence error rate (SER).

\section{Summary of Methodology}
In this chapter we outline how this research set about to achieve its objectives.  The main claim of this research is that by building a speech model that combines knowledge of end-to-end processing along with state of the art signal processing the overall complexity and time build new ASR systems can be reduced.

This chapter also reviews the technologies utilised by this research in order to arrive at the research outputs and briefly describes the experiments performed.  Within this space we describe CMUSphinx, Kaldi, Mozilla Deepspeech, Tensorflow, Matlab and Scatnet as major libraries used.   The first two being Hidden Markov Model (HMM)-based libraries and the latter being used either integrally or as part of end-to-end  and signal processing systems used to build Deep Recurrent Neural Network (RNN) models. Finally, we mention metrics for the evaluation of the models built in this work.


\chapter{Deep Scattering network}
Curve fitting is a very common theme in pattern recognition. The concept of invariant functions convey mapping functions that approximate a discriminating function when a parent function is reduced from a high dimensional space to a low dimensional space \cite{mallat2016understanding}.  In this chapter an invariance function called a scattering transform enables invariance of groups of deformations that could apply to speech signals thereby preserving higher level characterisations useful for classifying speech sounds. Works done by \citep{peddinti2014deep,zeghidour2016deep,anden2011multiscale,sainath2014deep} have shown that when the scattering spectrum are applied to speech signals and used as input to speech systems have state of the art performance.  In particular \cite{sainath2014deep} shows 4-7\% relative improvement in word error rates (WER) over Mel frequencies cepstral coefficients (MFCCs) for 50 and 430 hours of English Broadcast News speech corpus.  While experiments have been performed with hybrid HMM-DNN systems in the past, this thesis focuses on the use of scatter transforms in end-to-end RNN speech models.

This chapter iterates the use of the Fourier transform as the starting analysis function for building invariant functions and then discusses the Mel filter bank solution and then establishes why the scattering transform through the wavelet modulus operator provides better invariance features over the Mel filters.

\section{Fourier transform}
The Fourier transform often referred to as the power spectrum, allows us to discover frequencies contained within a signal.  The Fourier transform is a convolution between a signal and a complex sinusoid from $-\infty$ to $+\infty$ (Figure \ref{fig_4_1_fourier_eqn}). 

\begin{figure}
\centering
  % Requires \usepackage{graphicx}
  \includegraphics[width=7cm]{thesis/images/fourier.png}\\
  \caption{Fourier Equation} \label{fig_4_1_fourier_eqn}
\end{figure}
From the orthogonal property of complex exponential function, two functions are orthogonal if $\int f(x)g(x)=0$ where f(x) and g(x) are complementary functions, one being referred to as the analysis equation and the other referred to as the synthesis function.

If the discrete form of the Fourier transform analysis equation is given by
\begin{equation}
a_k=\frac{1}{T}\int_{-T/2}^{T/2}x(t)e^{\left(-j\frac{2\pi kt}{T}\right)}
\label{eqn_c4_fourier01}
\end{equation}

Then, the corresponding synthesis equation is given by
\begin{equation}
x(t)=\sum_{k=-\infty}^{\infty}a_ke^{\left(j\frac{2\pi kt}{T}\right)}
\label{eqn_c4_fourier02}
\end{equation}

Recall that $x(t)$ is the original signal while $a_k$ is the Fourier Series coefficient.  This coefficient indicates the amplitude and phase of the original signal's higher order harmonics indexed by $k$ such that higher values of $k$ correspond to higher frequency components.  In a typical spectrogram (figure \ref{fig_4_2_spectral}), it can be seen that the energy of the signal is concentrated about a central region and then harmonic spikes of energy content exponentially decrease and taper off.  Therefore in figure \ref{fig_4_2_spectral}, the energies are concentrated at frequencies of about 100, 150 and 400 hertz.
\begin{figure}
\centering
  % Requires \usepackage{graphicx}
  \includegraphics[width=7cm]{thesis/images/spectral.png}\\
  \caption{Sample Spectrogram} \cite{}\label{fig_4_2_spectral}
\end{figure}

The Fourier transform discussed in the preceding paragraph constitutes a valuable tool for the analysis of the frequency component of a signal.  However is not able to determine when in time a frequency occurs hence is not able to analyse time related signal deformations.  The Short-time Fourier Transform (STFT) attempts to salvage this by windowing the signal in time signal and performing Fourier transforms over sliding windows sections of the original signal rather than the whole signal.  There is however, a resolution trade off that ensues from this operation such that, the higher the resolution in time accuracy, the lower the frequency accuracy and vice versa.  In the next section on the continuous wavelet transform, how the wavelet transform improves on the weaknesses of the Fourier Transform and the STFT is reviewed.

\section{Wavelet transform}
The continuous wavelet transform can be defined as a signal multiplied by scaled and shifted version of a wavelet function $\psi(t)$ referred to as the mother wavelet. The time-frequency tile-allocation of the three basic transforms examined in the first part of this chapter is illustrated in figure \ref{fig_4_2_tftile}

\begin{figure}
\centering
  % Requires \usepackage{graphicx}
  \includegraphics[width=14cm]{thesis/images/tftile}\\
  \caption{Time frequency tiling for (a) Fourier Transform (b) Short-time Fourier Transform (STFT) (c) Wavelet transform}\label{fig_4_2_tftile}
\end{figure}

It can be seen here that for the Fourier transform there is no time information obtained.  In the STFT, as there is no way of telling where in time the frequencies are contained, the STFT makes a blanket range of the resolution of the window and is therefore equally tiled potentially losing information based on this setup.  For the case of the wavelet, because it is a scaled and shifted convolution, it takes care of the this problem providing a good resolution in both time and frequency.  The fundamental representation of the continuous wavelet function is:
\begin{equation}
C(a,b)=\int f(t)\frac{1}{\sqrt{a}}\psi\left(\frac{t-b}{a}\right)dt\label{eqn_c4_wavelet01}
\end{equation}
In this equation, $a$ and $b$ respectively represent the scaling and shifting resolution variables of the wavelet function. This is referred to as a mother wavelet. A few other mother wavelet functions discussed later in this chapter. Generally a mother wavelet is identified as being energy spikes in an infinite signal whose accumulative energy sums to zero.

\section{Discrete and Fast wavelet transform}
Synthesis and analysis equations (\ref{eqn_c4_fourier02} and \ref{eqn_c4_fourier01}) can be formulated as a linear combination of the basis $\phi_k(t)$ such that the basis, $\phi_k(t)=e^{j2\pi kt}$, and it's conjugate or orthonormal basis, $\tilde{\phi}_k(t)=e^{-j2\pi kt}$, equations (\ref{eqn_c4_fourier02} and \ref{eqn_c4_fourier01}) now become

\begin{equation}
x(t)=\sum_{k}a_k\phi_k
\label{eqn_c4_dwt02}
\end{equation}

\begin{equation}
a_k=\int x(t)\tilde{\phi}_k(t)
\label{eqn_c4_dwt01}
\end{equation}

With respect to scaling and shifting variables of continuous wavelet transforms in equation (\ref{eqn_c4_wavelet01}), a similar linear combination transformation can be applied by constructing orthonormal bases parameters, referred to as scaling ($\phi$) and translating ($\psi$) functions. For example, a simple Haar mother wavelet transform associated with a delta function, it is seen that:
\begin{equation}
\phi_{j,k}(t)=2^{j/2}\phi(2^jt-k)
\label{eqn_c4_dwt03}
\end{equation}
\begin{equation}
\psi_{j,k}(t)=2^{j/2}\psi(2^jt-k)
\label{eqn_c4_dwt04}
\end{equation}
where j is associated with the dilation (scaling) parameter and k is associated with the position (shifting) parameter. If the Haar coefficients $h_{(\cdot)}[n]=\{1/\sqrt{2},1/\sqrt{2}\}$ are extracted we have the following dilation and position parameters.
\begin{equation}
\phi(t)=h_\phi[n]\sqrt{2}\phi(2t-n)
\label{eqn_c4_dwt05}
\end{equation}
\begin{equation}
\psi(t)=h_\phi[n]\sqrt{2}\psi(2t-n)
\label{eqn_c4_dwt06}
\end{equation}

For any signal, a discrete wavelet transform in $l^2(\mathbb{Z})^1$ can be approximated by
\begin{equation}
f[n]=\frac{1}{\sqrt{M}}\sum_kW_\phi[j_0,k]\phi_{j_0,k}[n]+\frac{1}{\sqrt{M}}\sum_{j=j_0}^\infty\sum_kW_\psi[j,k]\psi_{j,k}[n]
\label{eqn_c4_dwt07}
\end{equation}
Here $f[n],\phi_{j_0,k}[n]$ and $\psi_{j,k}[n]$ are discrete functions defined in [0,M - 1], having a total of M points.  Because the sets $\{\phi_{j_0,k}[n]\}_{k\in\mathbf{Z}}$ and $\{\psi_{(j,k)\in\mathbf{Z}^2,j\ge j_0}\}$ are orthogonal to each other.  We can simply take the inner product to obtain the wavelet coefficients.
\begin{equation}
W_\phi[j_0,k]=\frac{1}{\sqrt{M}}\sum_nf[n]\phi_{j_0,k}[n]
\label{eqn_c4_dwt08}
\end{equation}
\begin{equation}
W_\psi[j,k]=\frac{1}{\sqrt{M}}\sum_nf[n]\psi_{j,k}[n] \quad j\ge j_0
\label{eqn_c4_dwt09}
\end{equation}
Equation (\ref{eqn_c4_dwt08}) is called approximation coefficient while (\ref{eqn_c4_dwt09}) is called detailed coefficients.

These two components show that the approximation coefficient, $W_\phi[j_0,k]$, models a low pass filter and the detailed coefficient,$W_\psi[j_0,k]$, models a high pass filter. It is possible to determine the approximation and detailed coefficients without the scaling and dilating parameters. The resulting coefficients, called the fast wavelet transform, are a convolution between the wavelet coefficients and a down-sampled version of the next order coefficients.  The fast wavelet transform was first postulated in \citep{mallat1989theory}.
\begin{equation}
W_\phi[j,k]=h_\phi[-n]\ast W_\phi[j+1,n]|_{n=2k, k\ge 0}
\label{eqn_c4_dwt10}
\end{equation}
\begin{equation}
W_\psi[j_0,k]=h_\psi[-n]\ast W_\phi[j+1,n]|_{n=2k, k\ge 0}
\label{eqn_c4_dwt11}
\end{equation}

For analysis of the Haar wavelet and the derivation of equations (\ref{eqn_c4_dwt10} and \ref{eqn_c4_dwt11}) see appendix \ref{app01}.

\section{Mel filter banks}

The Fourier and wavelet transform are general means of extracting information from continuous signals using the frequency domain and in the case of the Wavelet transform using both time and frequency domain.  The objective in machine learning, however, is to extract patterns from the derived information.  In this chapter, in particular, the Mel filter bank and the scatter transform are elaborated on as speech feature extractors.  They process high dimensional information obtained from the Fourier and wavelet transform signal processing techniques and reducing the information obtained as lower dimension features.  All this aimed towards loss-less encoding of speech signals relevant for speech recognition.

The Mel filter banks form the basis of the Mel Frequency Cepstral Coefficients (MFCCs) described by \citep{davis1980comparison}.  MFCCs are state-of-the-art speech feature engineering drivers behind automatic speech recognition acoustic models.  Other common speech features used in speech recognition include, Linear Prediction Coefficients (LPCs) and Linear Prediction Cepstral Coefficients (LPCCs),Perceptual Linear Prediction coefficients (PLP),  \citep{mcloughlin2009applied, dines2010measuring}.  The following paragraphs describe how the Mel filters are derived. 

The Mel scale as described by \cite{stevens1937scale} is a perceptual scale which measures sound frequencies as perceived by human subjects equidistant from a sound source as compared to the actual frequency.  This scale is non-linear as the human ear processes sound non-linearly both in frequency as well as amplitude.

For the case of frequency, the human ear can discriminate lower frequencies more accurately than the higher frequencies.  The Mel scale model this behaviour by utilising frequency bins.  The frequency bin ranges are narrow at low frequencies and become wide in higher frequencies. In the case of the speech signal amplitude, a similar process is observed where the ear discriminates softer sounds better than louder sounds.  Generally, sound will be required to be 8 times as loud for significant perception by the ear.  While the Mel scales handle the frequency non-linearity in the speech signal, the signal amplitude is linearised during feature extraction by taking the log of the power spectrum of the signal, also known as the cepstral values.  Furthermore, using a log scale also allows for a channel normalisation technique that employs cepstral mean subtraction. \citep{becchetti1999behaviour}.

The minimum frequency number of bins used for the Mel scale is 26 bins. In order to determine the frequency ranges we use the following formula to convert between the Mel scale and the regular frequency scale:
\begin{equation}
M(f)=1125\ln(1+f/700)
\label{eqn_c4_mel01}
\end{equation}
\begin{equation}
M^{−1}(m)=700\exp(m/1125)−1
\label{eqn_c4_mel02}
\end{equation}
A simple approximation for the Mel scale is obtained by applying linear scale for the first ten filters and for the first 1kHz of the speech frequency range then applying the following formula for the rest (Becchetti, 1999)\citep{becchetti1999behaviour}:
\begin{equation}
\Delta_m=1.2\times \Delta_{m-1}
\label{eqn_c4_mel03}
\end{equation}
where $m$ is the frequency bin index and  $\Delta_m$ is the frequency range between the start and end frequencies for the $m$-th bin. The resulting filters are overlapping filters shown in figure \ref{fig_4_3_mfilt}.
\begin{figure}
\centering
  % Requires \usepackage{graphicx}
  \includegraphics[width=14cm]{thesis/images/m-filters}\\
  \caption{Mel filter plot \citep{lyons_2012}}\label{fig_4_3_mfilt}
\end{figure}
For speech recognition, we compute a statistical value or coefficient for each Mel frequency bin from the inverse discrete fourier transform (IDFT) of the Mel filters.  The coefficients are also concatenated with their delta and delta-delta values.  The delta and delta-delta values are determined from the following equation:
 \begin{equation}
d_t=\frac{\sum_{n=1}^Nn(c_{t+n}−c_{t−n})}{2\sum_{n=1}^Nn^2}
\label{eqn_c4_mel04}
\end{equation}
where $c_x$ is the $x$-th coefficient and $2n$ is the delta range which is usually $2-4$. The delta values are first order derived coefficients obtained from the original Mel filter coefficients while the delta-delta values are second-order derived coefficients obtained from the first-order derived delta coefficients.

There are two reasons for obtaining the IDFT from the filter banks.  The first is that since the bins use overlapping windows, the filter bin outputs tend to be correlated and obtaining the IDFT helps to decorrelate the outputs.  Secondly, decorrelated signals optimise algorithm computation efficiency involving matrix operations such that rather than using full co variance matrix, it is much simpler to compute the matrix operations from the matrix diagonal.  Also note that for cepstral values obtained from taking the log of the power power spectrum, the discrete cosine transform (DCT) is used to obtain the IDFT.  This is as a result of the cepstral values being real and symmetric\citep{gales2008application}.

As an attempt for MFCCs to incorporate dynamic frequency changes of the signal, the deltas and the delta-deltas are obtained from the coefficient computation in equation \ref{eqn_c4_mel04}.  However, it is worthy to note that only the first 13 of the coefficients and the resulting dynamic coefficients are used as speech features as it is observed that higher frequency dynamic coefficients rather degrade ASR performance \citep{gales2008application}.

\section{Deep scattering spectrum}
In this section reference is made to \citep{anden2011multiscale, anden2014deep, zeghidour2016deep}. For a signal $x$ we define the following transform $W_x$ as a convolution with a low-pass filter $\phi$ and higher frequency complex analytic wavelets $\psi_{\lambda_1}$:
\begin{equation}
Wx=(x\star\phi(t),x\star\psi_{\lambda_1}(t))_{t\in\mathbb{R},\lambda_1\in\Lambda_1} \label{eqn_c4_dss01}
\end{equation}

We apply a modulus operator to the wavelet coefficients to remove complex phase and extract envelopes at different resolutions
\begin{equation}
|W|x=\left(x\star\phi(t),|x\star\psi_{\lambda_1}(t)|\right)_{t\in\mathbb{R},\lambda_1\in\Lambda_1} \label{eqn_c4_dss02}
\end{equation}
$S_0x=x\star\phi(t)$ is locally invariant to translation thanks to the time averaging $\phi$.  This time-averaging loses the high frequency information, which is retrieved in the wavelet modulus coefficients $|x\star\psi_{\lambda_1}|$.  However, these wavelet modulus coefficients are not invariant to translation, and as for $S_0$, a local translation invariance is obtained by a time averaging which defines the first layer of scattering coefficients
\begin{equation}
S_1x(t,\psi_{\lambda_1})=|x \star\psi_{\lambda_1}| \star\phi(t)\label{eqn_c4_dss03})
\end{equation}
It is shown in \cite{anden2014deep} that if the wavelets $\psi_{\lambda_1}$ have the same frequency resolution as the standard Mel-filters, then the $S_1x$ coefficients approximate the Mel-filter coefficients.  Unlike the Mel-filter banks however, there is a strategy to recover the lost information, by passing the wavelet modulus coefficients  $|x\star\phi_{\lambda_1}|$ through a bank of higher frequency wavelets $\psi_{\lambda_2}$:
\begin{equation}
|W_2||x\star\phi_{\lambda_1}|=\left(|x\star\psi_{\lambda_1}|\star\phi,||x\star\psi_{\lambda_1}|\star\psi_{\lambda_2}|\right)_{\lambda_2\in\Lambda_2} \label{eqn_c4_dss04})\end{equation}
This second layer of wavelet modulus coefficients is still not invariant to translation, hence we average these coefficients with a low-pass filter $\phi$ to derive a second layer of of scattering coefficients.
 \begin{equation}
|W_2||x\star\phi_{\lambda_1}|=\left(|x\star\psi_{\lambda_1}|\star\phi,||x\star\psi_{\lambda_1}|\star\psi_{\lambda_2}|\right)_{\lambda_2\in\Lambda_2}\label{eqn_c4_dss04})\end{equation}

Repeating these successive steps of computing invariant features and retrieving lost information leads to the scattering spectrum, as seen in Fig. 1, however speech signals are almost entirely characterized by the first two layers of the spectrum, that is why a two layers spectrum is typically used for speech representation. It is shown in [6] that this representation is invariant to translations and stable to deformations, while keeping more information than the Mel-filter banks coefficients
\begin{figure}
\centering
  % Requires \usepackage{graphicx}
  \includegraphics[width=14cm]{thesis/images/scatter.png}\\
  \caption{Scattering network - 2 layers deep} \cite{zeghidour2016deep}\label{fig_4_3_scatter}
\end{figure}
\label{ch4DSN}

\chapter{Wakirike Language Model}
A language model for the Wakirike language is developed in this chapter.  This model draws upon the premise that the grammar of a language is expressed in the character sequence pattern which is ultimately expressed in words and therefore the abstract grammar rules can be extracted and learned by a character-based RNN neural network.

\section{Data Preparation}
The Wakirike New Testament Bible served as the source of data for the deep neural network training.  As there wasn't a readily available soft or on-line copy of the Wakirike new testament bible for use, this was typed, giving rise to a complete corpus word size of 668,522 words and a character count of 6,539,176 characters. The dataset was then divided into 11 parts. Two parts dedicated for testing and validation and the remaining 9 parts were used for training.

The Unicode representations of the character set consisting of letters and punctuation marks are one-hot encoded and batched for sequential input, each batch having a character sequence length of 30 characters.


\section{GRU Training}.

The modified LSTM RNN known as the Gated Recurrent Unit (GRU) is employed for the neural network model in order to optimise network performance in terms of resource conservation.  GRUs have been shown to give similar performance to regular LSTMs however, with a lighter system resource footprint \citep{cho2014learning}. The GRU RNN used to train the Wakirike text corpus comprised an internal network size of 512 nodes for each layer and was 3 layers deep. Externally, 30 GRUs represented  the number of recurrent connections each connection representing a time step bearing contextual for the recurrent input sequence. 

To mitigate for overfitting, due to the multi-layered high-dimensional depth of this neural network, a small learning rate of 0.001 was used. To further marginalise overfitting the popular and effective dropout \citep{srivastava2014dropout} method for regularising deep neural networks kept at 20\% such that only 80\% activations were propagated from one layer to the next and the remaining 20\% were zeroed out.

\subsection{Output Language Generation}
The neural network is trained for 10 epochs and achieves a prediction accuracy of 85\% on held-out data.  With the GRU model it is possible to seed this network with an input character and select from the top-N candidates thus causing the Neural network to generate its own sentences.  In this scenario, the network is said to perform language generation by immanently constructing its own sentences.  The generated language was found to be intelligibly similar to that of the training data. 

A clever use of this new corpus generated by the GRU language model of this work was to determine a word-based perplexity metric for the GRU neural language model. In this work, the word-based perplexity metric was achieved from the output language generated by first estimating the word based perplexity on the training data.  The same perplexity calculation was then used on the generated neural language model corpus. The corpus size of the neural language model was made to be equivalent to that of the training data, that is containing 6,539,176 characters.  The perplexity calculation was based on a modified Kneser-Key 5-gram model with smoothing (Ref).  The results discussed below showed that the LSTM model generated a superior model compared to the n-gram model that better matched the training data.

The evaluation of the GRU language model of the Wakirike language was performed using a perplexity measurement metric. The Perplexity metric applies the language model to a test dataset and measures how probable the test dataset is. Perplexity is a relative measure given by the formula:


\begin{table}
  \caption{Perplexity Calculation results}
  \label{tab:example}
\begin{tabular}{lr}
\toprule
Language Model & Perplexity\\
\midrule
LSTM RNN & 2.6\\
3-gram with Keysner Soothing and interpolation & 3.3\\
\bottomrule
\end{tabular}
\end{table}


The result of the training of the Long-short-term-memory (LSTM)-Cell Recurrent Neural Network on low-resourced Wakirike Language gave impressive and intelligible results and showed better results when measured with standard n-gram language models. The results showed that it is indeed possible to use an LSTM on a low resource character sequence corpus to produce an Wakirike language model.

The evaluation of the LSTM language model of the Wakirike language was performed using a perplexity measurement metric. The Perplexity metric applies the language model to a test dataset and measures how probable the test dataset is. Perplexity is a relative measure given by the formula:

%
\begin{equation}
PP(W)=P(w_1,w_2\dots w_N)^\frac{1}{N}
\label{eq6}
\end{equation}
%
%
\begin{equation}
PP(W)=\sqrt[N]{\prod_{i=1}^N\frac{1}{P(w_i|w_{i-1})}}
\label{eq7}
\end{equation}
%

Where $w_1,\dots,w_N$ are the sequence of words. The language model with the lower relative perplexity score is therefore expected to yield better approximation of the data when applied to unseen data generally.

There was no way however to directly measure perplexity on a character sequence model because perplexity is usually used to evaluate word-based models.  However, this limitation was overcome by performing n-gram analysis on the corpus entirely generated from the LSTM network. The generated n-gram model from the generated corpus is then applied to test data and the perplexity is measured.

Table ~\ref{tab:example} below shows the Results of the Perplexity model of the LSTM Wakirike Language model and an equivalent Trigram Language model with interpolation and Keysner smoothing \cite{chen1996empirical}.
Table 1 below shows the Results of the Perplexity model of the LSTM Wakirike Language model and an equivalent Tri-gram Language model with interpolation and Keysner smoothing.

It can be inferred that the LSTM character-model developed has an improved language model and because it is based on a character-model, which is fine-grained when compared to a word model, it is likely to generalise data better when used in practice is and less biased than a word-based model.  This can be observed from the fact that the output corpus produced a larger vocabulary size.


\chapter{Deep Learning Speech Model}

\textcolor{blue}{
The details of the language models developed for the Wakirike language is discussed in this chapter.  The language models developed draw  upon the premise that the grammar of a language is expressed in the character sequence pattern ultimately rendered in word sequences.  The two models developed in this chapter follow RNN implementations discussed in chapter \ref{ch3RNN}. }

\section{Data Preparation}
\startblue
A published version of the Wakirike New Testament Bible was obtained and used as the data source for RNN training of the language model.  There was no readily available soft or on-line copy of the Wakirike new testament bible. As such, the Wakirike New Testament Bible text corpus text was entered into the ASR system from the physical copy using a text editor to form a text corpus.  The complete corpus had a word count of 668,522 words and a character count of 6,539,176 characters. Following the k-fold cross validation process, \citep{geron2019hands}, the data set was then divided into 11 parts. Two parts dedicated for testing and validation and the remaining nine parts were used for training. As the validation set is not seen during training it can be used to keep track of how well the training is going and that it is not over-fitting the data by simply memorising it.

Preprocessing of the text corpus involved selecting a set of characters as the input feature set and removing all other characters not found in the input feature set.  The Unicode representations of the character set consisted of letters and punctuation marks.  These are one-hot encoded and batched for sequential input.  Neural network parameters which are not automatically determined through back propagation are called hyper-parameters.  These are usually experimentally determined and manually set while configuring the network.  A hyper-parameter for the language model RNN is the input sequence length.  For the language model built a 30 characters-long sequence length is chosen.  This  length is an average phrase sequence.  In these phrases, long-term character dependencies of words can be captured. At the same time, keeping the sequence length at this value, and not longer, will pose less of a burden on the computer system resources during parameter computations.  

Another hyper-parameter for training used was the batch size.  The batch size parameter determined how many 30-character sequences will be trained in parallel in order to speed up the training process.  Increasing the batch size also meant an increase in the size of the matrix multiplications being performed and therefore, the computing power system resource being demanded by the language model.  By experimenting with various batch sizes it was determined a batch size of 200 was suitable for training the language model with respect to the other training parameters.

\section{GRU RNN Architecture}
The modified LSTM RNN known as the Gated Recurrent Unit (GRU) discussed in Chapter \ref{ch3RNN} is employed for the neural network model built in this Chapter.  In order to optimise network performance while conserving computation resources, GRUs have been shown to give similar performance to regular LSTMs however, with a lighter system resource footprint \citep{cho2014learning}. 

The architecture of the GRU RNN used to train the Wakirike text corpus had an internal network size of 512 nodes for each layer and was 3 layers deep. In a study by \citep{goodfellow2013multi}, it was shown that increasing the number of nodes in a neural network will lead to over-fitting, however, simultaneously increasing the network depth mitigates this effect.  In other words, in order to expand the degrees of freedom of a neural network and at the same time constrain the network to generalise well on unseen data, it is necessary to increase the number of neurons in both length and depth.  Experiments carried out in this chapter follow this recommendation. Initial experiments had an internal node size of 128 and a single layer deep.  While this showed promise of converging, the error rate was still high, therefore the network was expanded to the final model above.   Externally, the network model is further  sequenced 30 times, representing the input sequence length hyper-parameter and the number of recurrent connections where each connection represents a sequenced time step. 

Another hyper-parameter sensitive to network size is the learning rate.  The learning rate is selected in such a manner that an increase in the network size makes the learning rate more prone to overshooting.  Therefore, increased degrees of freedom in a neural network will require the learning rate to be made smaller so that it does not overshoot the network saturation point.  Small learning rates of between 0.001 and 0.005 were used. Furthermore, the language model neural network was designed to overcome over-fitting using the dropout (Srivastava et al., 2014) method which has been shown to be effective for regularising deep neural networks.  The hyper-parameter for dropout was kept at 20\% such that only 80\% of neural network activations are propagated from one layer to the next, whereas the remaining 20\% were randomly zeroed out.  Intuitively, dropout works by forcing the remaining active neurons to infer what is missing in the activations that have been dropped and ultimately leads to better generalisations as activations are based on inference than on memory.

\stopblue


\section{Output Language Generation}
The neural network was trained for 10 epochs and achieved a prediction accuracy of 85\% on held-out data.  With this GRU character-based language model, it is possible to seed this network with an input character and select from the top-N candidates thus causing the Neural network to generate its own sentences.  In this scenario, the network is said to perform language generation by immanently constructing its own sentences.  The generated language output from the GRU language model was found to be intelligible and a reflection of the overall context of the training data.

A clever use of this new corpus generated by the GRU language model of this work was to determine a word-based perplexity metric for the GRU neural language model. In this work, the word-based perplexity metric was determined from the output language generated by first estimating the word based perplexity on the training data.  The same perplexity calculation was then used on the generated neural language model corpus. The corpus size of the neural language model was made to be equivalent to that of the training data, that is containing 6,539,176 characters.  The perplexity calculation was based on a modified Kneser-Key 5-gram model with smoothing \citep{Heafield-estimate}.  The results discussed below showed that the GRU-based RNN model generated a superior model compared to the n-gram model that better matched the training data.

The evaluation of the GRU language model of the Wakirike language was performed using a perplexity measurement metric. The Perplexity metric applies the language model to a test data-set and measures how probable the test data-set is. Perplexity is a relative measure given by the formula:
%
\begin{equation}
PP(W)=P(w_1,w_2\dots w_N)^\frac{1}{N}
\label{ch5_eq1_ppx}
\end{equation}
%
%
\begin{equation}
PP(W)=\sqrt[N]{\prod_{i=1}^N\frac{1}{P(w_i|w_{i-1})}}
\label{ch5_eq2_ppx}
\end{equation}
%
Where $w_1,\dots,w_N$ are the sequence of words. The language model with the lower relative perplexity score is therefore expected to yield better approximation of the data when applied to unseen data generally.

The result of the training of the GRU-Cell Recurrent Neural Network on low-resourced Wakirike Language gave impressive and intelligible results and showed better results when measured with standard n-gram language models. The results showed that it is indeed possible to derive a language model using a GRU-cell RNN on a low resource character sequence corpus for the Wakirike language.

A character based perplexity metric is possible using  the negative log likelihood of the character sequence.
\begin{equation}
    PP(X)=exp\left\{−\frac{\sum_{t=1}^T\log P(x_t|x_{1:t−1})}{T}\right\}
\label{ch5_eq3_ppx}
\end{equation}

However, our base-line language model is a 5-gram word-based language model.  Therefore, comparing a word based model to a character based model requires a conversion step. In this work, the conversion step involved using the GRU language model generated a corpus which was rescored by re-estimating with a 5-gram word-based language model

Table ~\ref{tab:example} shows the Results of the Perplexity model of the LSTM Wakirike Language model and an equivalent 5-gram Language model with interpolation and Keysner smoothing \citep{Heafield-estimate} for various lengths of the held-out data.


\begin{table}
  \caption{Perplexity Calculation results}
  \label{tab:example}
\begin{tabular}{lrr}
\toprule
Language Model & Perplexity  \\
\midrule
Held-out data size (characters) & 998 & 99\\
\midrule
LSTM RNN & 1.6398 & 1.7622\\
5-gram with Keysner Soothing and interpolation & 1.8046 & 1.9461\\
\bottomrule
\end{tabular}
\end{table}
\section{Chapter Summary}
This chapter shows the application of a character-based Gated Recurrent Unit RNN on the low resource language of Wakirike to generate a language model for the Wakirike language.  The data-set and preparation and the details of the network were discussed.  The output of this model was used to hallucinate the Wakirike language which was then scored against word-based perplexity to obtain a metric against the baseline language model.

It can be inferred that the GRU character-model developed has an improved language model and because it is based on a character-model, which is fine-grained when compared to a word model, it is likely to generalise data better when used in practice and is less biased than a word-based model.  This can be observed from the fact that the output corpus produced a larger vocabulary size.


\chapter{Future study}
\startblue
Throughout the development of this thesis, the establishment of deep learning as a strategy where computers learn through representation of patterns at varying degrees of complexity has been an underlying theme.  It was also emphasised how this is achieved by internal layer-wise encapsulations. Structures discussed in Chapter \ref{ch2litrev}, such as layer-wise stacking of neural network type architectures such as the \acrfull{rbm} and \acrfull{dbn} were used to implement such representations.  

In this chapter, the end-to-end Bi-directional Recurrent Neural Network model is described.  \acrshort{birnn} for speech recognition tasks is  employed here as opposed to regular \acrshort{rnn}s or \acrshort{dbn}s mentioned above in the preceding paragraph.  \acrshort{birnn}s are used because of the contextual nature of speech.  In Chapter \ref{ch6_wlm} it was demonstrated  how deep stacking of \acrshort{gru}s outperform single-layer \acrshort{rnn}s for extended sequences. That is to say, words in a sentence or paragraph are contextual to the sentence/paragraph over particularly long sequences and, these word contexts are better captured by the GRU architecture.  More importantly, \acrshort{birnn}'s have a forward and backward \acrshort{rnn} and these give the neural network the ability to analyse (look-up) the words from the backward RNN not currently seen by the forward RNN in the sentence succinctly giving the BiRNN parameters a contextual feature \citep{graves2006connectionist}.  

In addition to the procedure for designing sequence-to-sequence \acrshort{rnn}s outlined in Section \ref{sec_341_rnnproc}, this Chapter describes the training data, data preprocessing, derivation of feature vectors and output decoding.  First, speech features developed by making use of the deep scattering convolution networks \acrshort{dsn} is discussed.   The \acrshort{dsn}s are used as inputs to the end-to-end model.  Two end-to-end networks are then described.  The core \acrshort{birnn} network and a second \acrshort{birnn} network augmented with an RNN-transducer and an attention mechanism. A formal presentation of the speech neural network model parameters and architecture is given and the decoding algorithm is also detailed in sections contained within this chapter.  Finally, the results are presented and the findings from the model results discussed.\stopblue

\section{Deep Scattering Features}\label{sec_c7_wparams}
\textcolor{blue}{In Chapter 4, we derived a fast wavelet transform from a low pass filter and a high pass filter.  The speech features used for the BiRNN is obtained from successive wavelet-modulus operations of a deep scattering network 2 layers deep.  This 2-layer \acrshort{dsn} comprises a first-order scatter transform. The wavelet modulus operator is derived from the combination of a low pass filter and a band pass filter}.  Hyper parameters of the system included the window period for each sampled sub section, $T$;  The Q-band value for the band pass filter and the number of wavelets $J$ at each scattering layer for the total number of layers, $M=2$.

The matlab scatnet toolbox \citep{anden2014scatnet}, used to determine the scatter coefficient features for this research, provides optimal values for hyper parameters for audio signal processing into scatter features.  In this regime the value for the hyper parameter $T=512$ samples per window. This corresponds to a window of $50$ milliseconds for the audio signals sampled at $8000 Hz$.  For the zeroth scattering layer the $Q$-band parameter was $Q=8$ and the first scattering layer took the value  $Q=1$.  Finally $J$ is pre-calculated based on the value of $T$.  These after Scat-Net processing produce a feature-vector having $165$ dimensions.  These feature vectors in turn are used as inputs to the bi-direction neural network model whose architecture is described in succeeding sections.

\textcolor{blue}{For the second end-to-end architecture involving a transducer with attention mechanism, a period of  is used to capture a window of 4 seconds for audio signals sampled at 16000Hz. The same Q-band parameters having Q=8 for the zeroth layer and Q=1 for successive layers are used, In addition, the total number of layers deep was M=3 giving rise to a 2nd-order Scatter Network. This produced a feature-vector having 250 dimensions.}

\section{CTC-BiRNN Architecture}\label{sec_c7_birnn}

\textcolor{blue}{The \acrshort{ctc}-\acrshort{birnn} sequence model design follows the synchronous \acrshort{mimo} design described in Section \ref{sec_postalign}. As a result of the CTC-decoder implementation in Section \ref{sec_c7_ctc_decoder}, however, the decoder converts \acrshort{birnn} model from a synchronous \acrshort{mimo} to an asynchronous one.}

The core of the system is a bidirectional recurrent neural network (BiRNN) trained to ingest scatter coefficients described in the previous section, in order to generate English text transcriptions.  An end-to-end system therefore specifies that utterances $x$ and the corresponding label $y$ be sampled from a training set such that the sample $S = {(x^{(1)}, y^{(1)}), (x^{(2)}, y^{(2)}), . . .}$.   In our end-to-end model, each utterance, $x^{(i)}$ is a processed feature vector consisting of $165$ dimensions.  Recall, every window passes through a scattering transform to yield an input of vector of $p=165$ features; consequently,   $x^{(i)}_{t,p}$ denotes the $p$-th feature in a scatter transform at time $t$.  

GPU training of the speech model architecture developed above was conducted using Mozilla Deepspeech \citep{mozilla_2019} CTC bi-directional RNN implementation along with the accompanying Mozilla Common voice data set  \citep{ardila2019common}.  The Common Voice Data set project consists of voice samples in short recordings approximately $4$ seconds each.  The complete data set is about $250$ hours of recording divided into training, test and development subsets.  The BiRNN, given the input sequence, $x$, outputs a sequence of probabilities $y_t=\mathbb{P}(c_t|x)$,  where $c_t \in a,b,c,\dots,z,space,apostrophe,blank$. 

The actual architecture of our core Bi-RNN is similar to the deepspeech system described in \cite{hannun2014deep}. This structure constitutes 5 hidden layers and one output layer.  The first three layers are regular DNNs followed by a bi-directional recurrent layer. As such, the output of the first three layers are computed by:
\begin{equation}
    h^{(l)}_t = g(W^{(l)} h^{(l−1)}_t + b^{(l)})\label{ch06_01_l1-3}
\end{equation}

$g(\cdot) = min\{max\{0,z\},20\}$  is the clipped rectified linear unit and $W^{(l)},b^{(l)}$ are weight matrix and bias parameters for layer  as described in sections \ref{dnn} and \ref{deepspeech} respectively.

It was shown in chapter \ref{ch3RNN} the recurrent layer comprise a forward and backward RNNs whose equations are repeated here for reference
\begin{equation}
    h^{(f)}_t = g(W^{(4)} h^{(3)}_t + W^{(f)}_r h^{(f)}_{t−1} + b^{(4)})
    \label{ch06_02_fwd}
\end{equation}
\begin{equation}
h^{(b)}_t = g(W^{(4)} h^{(3)}_t + W^{(b)}_r h^{(b)}_{t+1} + b^{(4)})    \label{ch06_03_bwd}
\end{equation}

Consequently, $h^{(f)}$ is the sequential computation from $t=1$ to $t=T^{(i)}$ for the $i$-th utterance and $h^{(b)}$ is the reverse computation from $t=T^{(i)}$ to $t=1$.  In addition the output from layer five is summarily given as the combined outputs from the recurrent layer:
\begin{equation}
h^{(5)} = g(W^{(5)} h^{(4)} + b^{(5)})    \label{ch06_04_l5}
\end{equation}
where $h^{(4)} = h^{(f)} + h^{(b)}$. The output of the Bi-RNN on layer 6 is a standard soft-max layer that outputs a predicted character over probabilities for each time slice $t$ and character $k$ in the alphabet:
\begin{equation}
h^{(6)}_{t,k} = \hat{y}_{t,k} \equiv \mathbb{P}(c_t = k \mid x) = \frac{\exp{ \left( (W^{(6)} h^{(5)}_t)_k + b^{(6)}_k \right)}}{\sum_j \exp{\left( (W^{(6)} h^{(5)}_t)_j + b^{(6)}_j \right)}})    \label{ch06_05_l6}
\end{equation}

$b^{(6)}_k$ takes on the -th bias and $(W^{(6)} h^{(5)}_t)_k$ is the matrix product of the $k$-th element.  The error of the outputs are then computed using the CTC loss function \cite{graves_2014} as described in chapter \ref{ch3RNN}.  A summary of our model is illustrated in Figure \ref{fig_6_1_ctc_scatter}.
\begin{figure}
\centering
  % Requires \usepackage{graphicx}
  \includegraphics[width=14cm]{thesis/images/ctc_scatter.png}\\
  \caption{Deep scattering Bi-RNN Model} \label{fig_6_1_ctc_scatter}
\end{figure}

\subsection{CTC Decoding} \label{sec_c7_ctc_decoder}

In chapter three the CTC loss function algorithm was established as being able to maximise the probability of two cases.  The first case of transiting to a blank and the second case of transiting to a non blank.  In this section, this concept is used to enable decoding of the network output from posterior distribution output to character sequences which can be measured against a reference transcription using either character error rate (CER) or word error rate (WER).

Recall, all the output symbols are in the alphabet $\Sigma$ and augmented with the blank symbol. The posterior output of the CTC network is the probability of the symbol given the speech feature input $p(c|x_t)$ at time $t$ for $t=1,\dots,T$ and $T$ is the length of the input sequence.  Also recall two further sets of probabilities also being maintained by the model are the probability of a blank character $p_b$ and that of a non blank character $p_{nb}$.

Several strategies have been employed to obtain a translation string from the output of the deep neural network.  The prefix beam search employed by the CTC decoder of this research is derived from an initial greedy approximation, where at each time step determine the argument that maximises the  probability $p(c|x_t)$ at each time step. Let $C=(c_1,\dots,c_T$ be the character string then, the greedy approach has 
\begin{equation}
    c_t=arg\max_{c\in\Sigma}p(c|x_t)
\end{equation}
However, this simple approximation is unable to collapse repeating sequences and remove blank symbols. In addition, the approximation is unable to include the constraint of a lexicon or language model.

The prefix beam search algorithm \cite{hannun2014first} adopted in this work incorporates a language model derived from a lexicon in addition to keeping track of the various likelihoods used for decoding.  For the language model constraint, the transcription $W$ is recovered from acoustic input $X$ at time $t$ by choosing the word which maximising the posterior probability:
\begin{equation}
W_i=arg\max_{W_i \in \Sigma_W} p_{net}(W;X)p_{lm}(W)
\label{eqn_c6_decoder01}
\end{equation}
In equation \ref{eqn_c6_decoder01}, the Bayes product of language model prior $p_{lm}$ and the network output $p_{net}$ are utilised to maximise the probability of a particular character-word sequence in the lexicon given by $\Sigma_W$.  The overall calculation used to derive the final posterior distribution includes word insertion factors ($\alpha$ and $\beta$) used to balance the highly constrained n-gram language model.

The second strategy adopted by the prefix beam search which improves the decoding algorithm is the beam search strategy.  With this approach, the search maintains all possible paths; however, it retains only $k$ number paths which maximise the output sequence probability.  Improvements gained with this method are seen when certain maximal paths are made obsolete owing to new information derived from the multiple paths in being maintained in memory. 

The recursive prefix beam search algorithm illustrated in Figure \ref{fig_c6_decoder01} attempts to find the string formulated in equation \ref{eqn_c6_decoder01}.  Two sets prefixes $A_{prev}$ and $A_{nxet}$ are initialised, such that at $A_{nxet}$ maintains the prefixes in the current time-step while $A_{prev}$ maintains only $k$-prefixes from the previous time-step.  Note that at the end of each time step $A_{prev}$ is updated with only -most probable prefixes from $A_{nxet}$. Therefore while,  $A_{nxet}$ contains all the possible new paths from based on $A_{prev}$ as a Cartesian product of $A_{prev} \times \Sigma \in \mathcal{Z}^k\times\mathcal{Z}^{|\Sigma|}$ where $|\Sigma|$ is the length of $\Sigma$. The probabilities of each prefix obtained at each time step are the sum of the probability of non-blank plus the probability of a blank symbol.
\begin{sidewaysfigure}[ht]
    \includegraphics[width=22cm]{ctc}
    \caption{Prefix beam search algorithm}
    \label{fig_c6_decoder01}
\end{sidewaysfigure}

At every  time step and for every prefix $\ell$ currently in $A_{prev}$, a character from the alphabet $\Sigma$ is presented to the prefix. The prefix is only extended only when the presented symbol is not a blank or a space. $A_{nxet}$ and $A_{prev}$ maintain a list of active prefixes at the previous time step and proposed prefixes at the next time step respectively, The prefix probability is given by multiplying the word insertion term by the sum of the blank and non-blank symbol probabilities.
\begin{equation}
p(\ell|x_{1:t})=(p_{nb}(\ell|x_{1:t})+p_b(\ell|x_{1:t}))|W(\ell)|^\beta
\label{eqn_c6_decoder03}
\end{equation}

$W(\cdot)$ is obtained by segmenting all the characters in the sequence with the space-character symbol and truncating any characters trailing the  set of words in the sequence.  The prefix distribution however varies slightly depending on network output character being presented.

$\ell_{end}$ is the variable representing the last symbol in the prefix sequence in $A_{prev}$. If the symbol presented is the same as $\ell_{end}$ then the probability of a non-blank symbol,$p_{nb}=0$ . If the symbol being presented is blank then we do not extend the prefix.  Finally, if the symbol being presented is a space then we invoke the language model as follows
\begin{equation}
p(\ell^+|x_{1:t})=p(W(\ell^+)|W(\ell))^\alpha(p_{nb}(\ell|x_{1:t})+p_b(\ell|x_{1:t}))|W(\ell)|^\beta
\label{eqn_c6_decoder03}
\end{equation}

Note that $p(W(\ell^+)|W(\ell))$ is set to $0$ if the current word $W(\ell^+)$ is not in the lexicon. This becomes a constraint to enforce all character strings to consist only of words in the lexicon.  Furthermore,  $p(W(\ell^+)|W(\ell))$ is extended to include all the character sequences representing number of words considered by the n-gram language model by constituting the last $n-1$ words in character sequence $W(\ell)$.

\subsection{Model Hyper parameters}
The hidden layer matrix for each layer comprised 1024 hidden units (6.6M free parameters).  The weights are initialised from a uniform random distribution having a standard deviation of 0.046875.  The Adam optimisation algorithm \citep{kingma2014adam} was used with initial learning rate of, and a momentum of 0.95 was deployed to optimise the learning rate.

The network was trained for a total of five to fifty epochs over the training set for experiments conducted. The training time for Python GPU implementation is shown in Table \ref{tab_c6_01_training}.  For decoding with prefix search we use a beam size of $200$ and cross-validated with a held-out set to find optimal settings for the parameters $\alpha$ and $\beta$. Figure \ref{fig_6_3_wer} shows word error rates for various GPU configurations and audio data-set sizes.

\startblue
\section{Summary of \acrshort{birnn} Experiment Design}
The details of the \acrshort{birnn} model has been outlined in Sections \ref{sec_c7_birnn} and \ref{sec_c7_ctc_decoder}.  This section now summarises the process of the \acrshort{birnn} experiment from data collection to output text transcriptions.  Recall once again, as this is an end-to-end experiment the input-end will comprise raw audio speech utterances and at the output-end will be the character sequences which are resolved into words using a language model. The design of this experiment therefore utilises the \acrshort{birnn} to ingest raw audio utterances as preprocessed scatter-transforms and outputs text transcription which can be compared against the original audio transcriptions.

The audio clips and the corresponding transcriptions are downloaded with a script and the subsequent locations are stored into a configuration file. Being an end to end process, no further data pre-processing is required except conversion of the audio file from a binary format to a numeric-text format.  From this numeric text format, the scatter-transforms are computed when loaded from the configuration file.  

The \acrshort{ctc}-algorithm discussed in Sections \ref{sec_c4_ctcloss} and \ref{sec_c7_ctc_decoder} is responsible for correcting fuzzy alignments between audio input and output text.  This relationship according to Section \ref{sec_341_rnnproc} is an asynchronous \acrshort{mimo}, but the \acrshort{birnn} represents a synchronous \acrshort{mimo}. Hence a synchronous \acrshort{mimo} is combined with the CTC-decoder such that while \acrshort{birnn} takes care of the \acrlong{mimo} relationship the \acrshort{ctc} decoder takes care of collapsing output characters and, therefore, restoring the asynchronous relationship between outputs and inputs.  For steps 1 and 2 of Section \ref{sec_341_rnnproc} we use a \acrshort{birnn}.  Details of the internal structure of the \acrshort{birnn} from Sections \ref{sec_c7_birnn} and \ref{sec_c7_ctc_decoder} are used for step 3.  This includes 3 hidden regular \acrshort{dnn} layers and two bi-directional \acrshort{lstm}s hidden-layers and a softmax output layer.  Neural network saturation parameters are selected based on desired training time, similar research practice, and accuracy expectations in line with the research objectives. They include weight initialization having a mean of 0 and standard deviation of 0.046875; clipped \acrshort{relu} non-linear function (see Chapter \ref{ch4DSN}); learning rate of 0.001 with 0.95 momentum.  Finally, the adam optimiser and CTC-loss function (Section \ref{sec_c4_ctcloss}) are incorporated at the training output stages.

The \acrshort{ctc} decoder described in Section \ref{sec_c7_ctc_decoder} is then used to determine the characters from the softmax output.  The \acrshort{ctc} decoder had a look ahead beam search parameter of 200 characters.

\section{BiRNN with Attention Transducer end-to-end Architecture}\label{sec_7_5_blstm_t}

The core of this model is a CTC-Transformer+Attention Transducer model.  Together these two architectures achieve joint speech training and decoding.  The CTC-Transformer model is based on a Bi-LSTM similar to what is obtainable in the DeepSpeech model. There are up to 11 variants of Attention networks implemented in \acrshort{espnet}, however, the results of the experiments done this work experiment was determined from the attention model described in \cite{chorowski2015attention}.  Moreover, the multi-objective training was performed with equal weights on both the CTC-transformer and the Attention-Transducer.  Finally the system was trained for 20-200 epochs depending on the design goals and accuracy required.

This model uses the asynchronous \acrshort{mimo} model.  Although this model would require more neural network layers and about 2 times more \acrshort{rnn} units than the synchronous \acrshort{mimo}-\acrshort{rnn} model, ultimately, addition of Attention-based models ensures faster convergence and ultimately faster time to train.  This can only be achieved using asynchronous \acrshort{mimo} models which are the only models that support attention-mechanism.

Using a weighting function, $\alpha$, one can control how much bias either the CTC-Transform or the Attention-Transducer will get during training.  The joint training helps to improve robustness as well as achieve fast convergence.

\begin{equation}
    \mathcal{L}=\alpha\mathcal{L}^{ctc}+(1-\alpha)\mathacal{L}^{att}
    \label{eqn_c7_esp00}
\end{equation}

At the same time joint decoding of labels is integrated with the character based RNN-language modelling. The log probability of the RNNLM-integrated decoding of character labels is as follows

\begin{equation}
    \log p(y_n|y_{1:n−1},\mathbf{h}_{1:T‘})=\log p^{hyp}(y_n|y_{1:n−1},\mathbf{h}_{1:T‘})+\beta\log p^{lm}(y_n|y_{1:n−1})
    \label{eqn_c7_esp01}
\end{equation}
Where joint decoding, $\log p^{hyp}(y_n|y_{1:n−1},\mathbf{h}_{1:T‘})$ is given by
\begin{equation}
    \log p^{hyp}(y_n|y_{1:n−1},\mathbf{h}_{1:T‘})\\
    & = \alpha\log p^{ctc}(y_n|y_{1:n-1},\mathbf{h}_{1:T'})+(1-\alpha)\log p^{att}(y_n|y_{1:n-1},\mathbf{h}_{1:T'})
\end{equation}


\section{Summary of \acrshor{birnn} with Attention Transducer Experiment Design}
According to the stepwise procedure of deriving \acrshort{rnn} sequence-to-sequence models , the \acrshort{birnn} with Attention Transducer model incorporates an asynchronous \acrshort{mimo} design for Step 1 (Section \ref{sec_341_rnnproc}).  For steps 2 and 3 (Section \ref{sec_341_rnnproc}), 6 layers similar to the \acrshort{birnn}-only design consisting 3 regular \acrshort{dnn} layers, 2 \acrshort{blstm} and the output softmax layer. Neural network components  The number of neurons for the hidden layers were 2048 neurons.  Network saturation parameters were similar to the \acrshort{birnn}-only experiment setup having the number of epochs between 20-200 epochs and in line with research objectives.  As the training was significantly faster in this setup, more epochs could be incorporated into the experiments.  The initial experiment had only 20 epochs using baseline experiments and to determine how fast the training would be.  Subsequent experiments were between 100 and 200 epochs according to desired accuracy and research objectives.  Weights were initialized between 1 and -1 according to a normal distribution.  Learning rate was 0.001; Non linear function was a clipped \acrshort{relu}.  Cost function and optimizer was ctc-loss and ada-grad variant. Decoding was done according to the joint decoding function described in Section \ref{sec_7_5_blstm_t}.  Finally, a drop out of 10\% was used to avoid over fitting of the model.

\section{Speech Model Baselines}
The model baselines were trained alongside their scatter transform counterparts.  In addition, we adopted the model produced by the Mozilla DeepSpeech team.  This model had a similar architecture with 5 hidden units and 2048 hidden units.  This baseline was trained on Librespeech corpus and the common voice data corpora \citep{panayotov2015librispeech, ardila2019common}.   Study by \cite{hannun2014first} reported successful character error rate (CER)  using deep neural network (DNN), recurrent deep neural network with only forward temporal connections (RDNN), and also bi-directional recurrent neural networks (BRDNN). The models used in their study had 5 hidden layers having either 1,824 or 2,048 hidden units in each hidden layer.  

Word Error Rates obtained by this additional model were optimised after 75 epochs, learning rate of 0.0001 and a dropout rate of 15\%.  In addition, the language model hyper parameters for alpha and beta were 0.75 and 1.85 respectively.  This achieved 8\% WER. This model was developed using MFCC features of the training corpora.

\section{Speech Model Simulations}

Speech model training experiments were carried out on the two different end-to-end models as well as on different \acrshort{gpu} configurations. The first set of experiments was performed for the \acrshort{birnn}-only model.  The first \acrshort{gpu} configuration for the \acrshort{birnn}-only model consisted of 2 \acrshort{gpu}s having a total of 10 gigabytes of memory. The second \acrshort{gpu} configuration comprised 5 \acrshort{gpu}s having a total of 15 gigabytes of memory. Experiments for the BiRNN end-to-end model with transducer and attention were also performed using a \acrshort{gpu} configuration having 4 gigabytes of memory. 

\subsection{\acrshort{birnn}-only end-to-end model Experiments}
For the \acrshort{birnn}-only end-to-end experiments, GPU configuration experiments were carried out on varying-size subsets of the common voice corpus.   The various GPU configurations along with the training times are shown in Table \ref{tab_c6_01_training}.

\begin{table}
  \caption{\acrshort{birnn}-only Experiments}
  \label{tab_c6_01_training}
\begin{tabular}{lccc}
\toprule
Experiment & Hours of speech & Total training time & Training status\\
\midrule
1. 2xGPU 10GB RAM & 1 & 7 days & Complete\\
2. 2xGPU 10GB RAM & 10 & 40 days & Not complete\\
3. 5xGPU 15GB RAM & 2 & 2 days & Complete\\
4. 5xGPU 15GB RAM & 40 & 40 days & Not complete\\
5. 1xCPU 16GB RAM & 20 & 20 days & Not complete\\
6. 1xGPU 2GB RAM & 20 & 20 days & Not complete\\
\bottomrule
\end{tabular}
\end{table}

It can be seen in Table \ref{tab_c6_01_training}, only two experiments had reached completion.    The others had to be stopped for exceeding reasonable training times of 20 and 40 days. Out of the four experiments that did not complete, all the GPU-based experiments had trained for up to 20 epochs and quantitative metrics were taken for these experiments.  Table \ref{tab_c6_02_training} shows the details for the \acrfull{wer} accuracy metrics for a total of four experiments. The number of hours of speech, corpus type and total number of epochs are also shown. Accuracy curves are shown in figure \ref{fig_6_3_wer}.

\begin{table}
  \caption{\acrshort{birnn}-only Experiments Summary}
  \label{tab_c6_02_training}
\begin{tabular}{lccccc}
\toprule
Experiment & Hours & Corpus & epochs & Metric & Score\\& of speech\\
\midrule
1. 2xGPU 10GB RAM & 1 & CV LVCSR & 40 & WER(\%) & 100+\\
2. 2xGPU 10GB RAM & 10 & CV LVCSR & 25 & WER(\%) & 100+\\
3. 5xGPU 15GB RAM & 2 & CV LVCSR & 40 & WER(\%) & 100\\
4. 5xGPU 15GB RAM & 40 &  CV LVCSR & 40 & WER(\%) & 87\\
\bottomrule
\end{tabular}
\end{table}


\begin{figure}
\centering
  % Requires \usepackage{graphicx}
  \includegraphics[width=14cm]{thesis/images/brnn_only.png}\\
  \caption{\acrshort{birnn}-only Experiments Error curve, where w\<x\<y\<z are taken arbitrarily across the total
number of epochs} \label{fig_6_3_wer}
\end{figure}


\subsection{\acrshort{birnn} with Attention Transducer Experiments}
The \acrfull{espnet} \citep{watanabe2018espnet} provides building blocks for \acrshort{blstm} transducer with attention mechanism  described in Section \ref{sec_7_5_blstm_t}.  Two experiments involving a much smaller audio corpus guaranteed to converge quickly at training and a larger Italian language speech corpus \citep{foxvorge2019} used for these experiments.  The AN4 (alphanumeric) corpus by Carnegie Mellon University \citep{acero1990acoustical}, is a small vocabulary speech corpus having only 948 training utterances and 140 test utterances.

The speech corpora utterances are 16-bit linearly sampled at 16kHz, each recording made with near-field microphone quality.  The compressed tar file comes with a variety of audio formats including raw wav format, the NIST sphere format and those already encoded as Mel cepstral coefficients.

The end-to-end architecture at the core of ESPNet is the CTC-Transformer+Attention Transducer model.  Together these two architectures achieve joint multi-objective speech training and decoding.  The CTC-Transformer model is based on a \acrshort{blstm} and is similar to what is obtainable in the DeepSpeech model.  There are up to 11 variants of Attention networks implemented in \acrshort{espnet}, however, the results of the \acrshort{espnet} experiment performed was determined from the model described in \cite{chorowski2015attention}.  Moreover, the multi-objective training was performed with equal weights on both the CTC-transformer and the Attention-Transducer.  

\begin{table}
  \caption{\acrshort{birnn} with attention and transducer Experiments}
  \label{tab_c6_03_training}
\begin{tabular}{lcccc}
\toprule
Experiment & Hours & Training  & Epochs & Training \\
& of speech & time & & status \\
\midrule
1. 1xGPU 4GB (log mel.) & 1 & 15 minutes & 20 & Complete\\
2. 1xCPU 16GB (scatter feat) & 1 & 3 days & 100 & Complete\\
3. 1xGPU 4GB (log mel) & ~10 & 11 hours & 200 & Complete\\
4. 1xGPU 4GB (scatter feat) & ~10 & 38 hours & 200 & Complete\\
\bottomrule
\end{tabular}
\end{table}
Experiments were carried out using ESPNet default parameters which included those for character based-Recurrent Neural Network language model RNN-LM, multi-channel feature input and multi-objective learning using both CTC-Transformer and Attention-Transducer networks.

With this minimal default setting, the test set had a final recognition score of 9.5\% character error rate (CER).  The next Chapter discusses how the baseline can be scaled and remodelled for integrating scattering features.

\begin{table}
  \caption{\acrshort{birnn} with attention and transducer Experiments Summary}
  \label{tab_c6_04_training}
\begin{tabular}{lcccc}
\toprule
Experiment & Hours & Corpus & Metric & Score\\& of speech\\
\midrule
1. 1xGPU 4GB (log mel) & 1 & AN4 SVCSR & WER(\%) & 12.9\\
2. 1xGPU 4GB (scatter feat.) & 1 & AN4 SVCSR & WER(\%) & 26.8 \\
3. 1xGPU 4GB (scatter feat) & ~10 & Voxforge-italian (LVSCR) & WER(\%) & 76.7 \\
4. 1xGPU 4GB (log mel) & ~10 &  Voxforge-italian (LVSCR) & WER(\%) & 72.4\\
\bottomrule
\end{tabular}
\end{table}

\section{Model Results Interpretation}
Interpretation of Model Results
Experiments carried out to train the end-to-end \acrshort{asr} were performed on the following system configurations
\begin{enumerate}
    \item GPU (GTX1050) with 2GB RAM
    \item GPU (GTX1050) with 4GB RAM
    \item GPU (GTX1060) with 6GB RAM
    \item GPU (GTX1070) with 3GB RAM
    \item CPU with 16GB RAM
\end{enumerate}

\subsection{Bi-RNN-only experiment discussion}
Configurations with CPU were used as control experiments to compare the GPU efficiency with the CPU being compensated with more memory.  The higher memory allowed the CPU configurations to remain accessible during training unlike the GPU systems that used up most of the system resources bringing the computer system close to a grinding halt and making the GPU systems difficult to access while training was in progress.  In as much as a number of the experiments done exceeded 10 days, our goal was not to exceed more than 10 days training for speech models.  What the GPU lacked in memory resources was compensated for in computational speed gained due to their capacity for parallel.  By changing the batch size, memory resource requirement and computational parallelism was simultaneously managed for all experiments.  Therefore for CPU training computational speedup was attempted by increasing the batch size and for GPU training batch sizes were reduced so as not to quickly deplete the small memories.  For \acrshort{birnn}-only experiments, regardless of the batchsize allocations, only 2GPU configurations  (1 and 3 in Table \ref{tab_c6_01_training}) completed training for the given amounts of epochs and training data.

Table \ref{tab_c6_02_training} shows four GPU-only configurations.  These GPU training configuration experiments had completed at least 20 epochs.  Training metrics for these configurations are plotted in Figure \ref{fig_6_3_wer}.   A reduction in training loss is observed once the data was increased to two hours of training.  This gives an indication of the model learning on the amount of data given.  Even though the speech models were trained on English language only.  We can simulate low resource settings in the English language by limiting the amount of data available during training.  Moreover, word error rates (WER) only showed improvement on the 40 hours data set.  This also indicates that a threshold of about 40 hours is required for the model to begin to converge for a Large Scale Vocabulary Continuous Speech Recognition (LSVCSR) system

The results showed that the training of the model was moving towards a very slow convergence as indicated by the slow decrements in training loss.  Initial experiments were performed on single GPU Units.  Batch size settings for these experiments were very small to fit into the limited RAM sizes on the GPUs.  At a later stage, multi-GPU units were utilised as a strategy to speed up training by increasing the batch sizes to run on the combined memory.  This however did not result in the anticipated speed up. It is suspected that this outcome may have been as a result of latency copying model parameters between GPU units and CPU multiple times during training.

\subsection{Bi-RNN with Transducer and attention mechanism experiment discussion}
Results from Bi-RNN with Transducer and attention experiments had shown greater promise in terms of completion of training within the time constraints set. Results shown in table \ref{tab_c6_03_training} show that both CPU and GPU training completed training for less than 20 hours of training and total number of epochs.

We used a \acrfull{svcsr} corpus of English language and a \acrfull{lvcsr} of Italian and achieved a decent score of 26.8\% for the \acrshort{svcsr} corpus and a high error score of 76.7\% for the \acrshort{lvcsr} corpus.  The high error score for the \ref{lvcsr} from the observed results was attributed to the fact that the amount of data given ~10 hours is not sufficient for meaningful convergence.  This is also evidenced by the baseline result having a similar high error rate of 72.4\%.  A similar effect was also observed in the Bi-RNN-only experiments such that after training of 40 hours of data for 40 epochs the error was at 87\%.

From the training curves (Figures \ref{fig_6_2_loss} and \ref{fig_6_3_wer} however, we can see that at 40 epochs the Bi-RNN with transducer and attention mechanism experiments had a faster rate of convergence and this led to experiments being completed within the time limits set.

\section{Chapter Summary}

In this chapter the details of the combination of an end-to-end deep bi-RNN architecture and deep scattering features were elaborated on.  The architecture described follows a five-layer structure consisting of a feed-forward neural network in the first three layers and the last two consisting of recurrent structures flowing in two different directions.  A 163-dimension 1st-order feature vector of deep-scattering encoding derived from a sampled raw audio file is fed into this network.

A second set of experiments comprising a similar architecture containing a \acrshort{bilstm} this time with encoder and decoder architectures and a transducer is also developed and tested.

The result showed the second set of experiments having the transducer architecture with attention mechanism are able to train faster but out of all the results, although some models came close to their respective baselines none actually performed better than the baseline models.  In Chapter \ref{ch8_future} we address ways that the results may be improved.

\stopblue

\begin{figure}
\centering
  % Requires \usepackage{graphicx}
  \includegraphics[width=14cm]{thesis/images/scatter_res.PNG}\\
  \caption{Training Loss, where $w<x<y<z$ are taken arbitrarily across the
total number of epochs} \label{fig_6_2_loss}
\end{figure}



\spacing{1.0}
\bibliographystyle{plainnat}

\bibliography{bib}

\end{document}
